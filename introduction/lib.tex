\chapter{Introduction} \label{chap:intoroduction}

% the code below specifies where the figures are stored
\ifpdf
    \graphicspath{{1_introduction/figures/PNG/}{1_introduction/figures/PDF/}{1_introduction/figures/}}
\else
    \graphicspath{{1_introduction/figures/EPS/}{1_introduction/figures/}}
\fi

\begin{chapquote}{Unknown.}
``If Bitcoin was the calculator, Ethereum was the ENIAC\footnote{the first generation computer
developed in 1944. It fills a 20-foot by 40-foot room and has 18,000 vacuum tubes.}. It is
expensive, slow and hard to work with. The challenge of today is to build the \textbf{commodity,
accessible and performant} computer.''
\end{chapquote}


Blockchains are indeed an interesting topic in 2020. Many believe that they are a revolutionary
technology that will shape our future societies, much like the internet and how it has impacted many
aspects of how we live in the last few decades \cite{will_blockchain_be_big_deal}. Moreover, they are
highly sophisticated and inter-disciplinary software artifacts, achieving high levels of
decentralization and security, which was deemed impossible so far. To the contrary, some people
skeptically see them as controversial, or merely a "hyped hoax", and doubt that they will ever
deliver much real value to the world. Nonetheless, through the rest of this chapter and this work
overall, we provide ample reasoning to justify why we think otherwise.

In very broad terms, a blockchain is a tamper-proof, append-only ledger that is being maintained in
a decentralized fashion, and can only be updated once everyone agrees upon that change as a bundle
of transactions. This bundle of transactions is called a \textbf{block}. Once this block is agreed
upon, it is appended (aka. \textit{chained}) to the ledger, hence the term block-\textit{chain}.
Moreover, the ledger itself is public and openly accessible to anyone. This means that everyone can
verify and check the final state of the ledger, and all the transactions and blocks in its past that
lead to this particular ledger state, to verify everything. At the same time, asymmetric cryptography
is the core identity indicator that one needs to interact with the chain, meaning that one's
personal identity can stay private in principle, if that is desired based on the circumstances. For
example, one can share the proof of owning some data, without actually sharing the data itself,
typically known as zero knowledge proofs \cite{Goldreich_Oren_1994_ZK}.

In some sense, blockchains are revolutionary because they remove the need for \textit{trust}, and
release it from the control of one entity (i.e. a single bank or institute), by encoding it as a
self-sovereign decentralized software. Our institutions are built upon the idea that they manage
people's assets, matters and belongings, and they ensure veracity, because we trust in them. In
short, they have \textbf{authority}. Of course, this model could work well in principle, but it
suffers from the same problem as some software do: it is a \textbf{single point of failure}. A
corrupt authority is just as destructive as a flaky single point of failure in a software is.
Blockchain attempts to resolve this by deploying software (i.e. itself) in a transparent and
decentralized manner, in which everyone's privacy is respected, whilst at the same time everyone can
still check the veracity of the ledger, and the software itself. In other words, no single entity
should be able to have any control over the system.

Now, all of these great properties do not come cheap. Blockchains are extremely complicated pieces
of software, and they require a great deal of expertise to be written correctly. Moreover, many of
the machinery used to create this \textit{decentralized} and \textit{public} append-only ledger
requires synchronization, serialization, or generally other procedures that are likely to decrease
the throughput at which the chain can process transactions. This, to some extent, contributes to the
skepticism about blockchains' feasibility. For example, Bitcoin, one of the famous deployed
blockchains to date, consumes a lot of resources to operate, and cannot execute more than around half a dozen transactions per second \cite{security_of_bitcoin}.

Therefore, it is a useful goal to investigate the possibilities through which a blockchain
system can be enhanced to operate \textit{faster}, and more \textit{efficiently}.

\section{Research Questions} \label{chap_intro:sec:resarch_q}

We have seen that blockchains are promising in their technology, and unique traits that they can
deliver. Yet, they are notoriously slow. Therefore, we  pursue the (initial) goal of improving
the \textit{efficiency} of a blockchain system. There are numerous ways to
achieve this goal, ranging from redesigning internal protocols within the blockchain to applying
concurrency. In this thesis, we precisely focus on the latter, enabling concurrency within
transactions that are processed and then appended to the ledger. Moreover, we do so by leveraging,
and mixing the best attributes of two different realms of concurrency, namely static analysis and
runtime conflict detection \footnote{By static we mean generally anything which is known at the
\textit{compile} phase, and by runtime the the \textit{execution} phase}. This approach is better
compared with other alternatives in Section~\ref{chap_design:sec:ways_to_speedup}. We also mention (in
the same section) why each of these approaches could have their own merit and value, and how they
differ with one another.

Based on this, we formulate the following as our research questions:

 \begin{enumerate}
     \item [RQ1] What approaches exist to achieve concurrent execution of transactions within a blockchain
     system?
	 \item [RQ2] How could both static analysis and runtime approaches be combined together to achieve a new approach with minimum overhead and measurable benefits?  
	 \item [RQ3] How would such an approach be evaluated against and compared to others?
 \end{enumerate}

\section{Thesis outline}
The rest of this thesis is organised as follows: \todo[inline]{write once the outline is done. This
is not an important paragraph anyhow.}

