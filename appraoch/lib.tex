\chapter{Approaches toward Concurrency} \label{chap:approach}

In this chapter we will build up all the details and arguments needed to introduce our approach
toward concurrency within blockchains. We will start with an interlude, enumerating different ways
to make blockchains achieve a higher throughput from an end-to-end perspective, at the end of which
we point out concurrency as our method of choice.

\section{Prelude: Speeding up a Blockchain - An Out of The Box
Overview}\label{chap_approach:sec:ways_to_speedup}

As mentioned, blockchains can be seen, in a very broad way, as a \textit{decentralized state machine
that transitions by means of transactions}. The throughput of a blockchain network, measured in
transactions per second, is a function of numerous components and can be analysed from different
points of view. While in this work we focus mainly on one aspect, it is helpful to enumerate
different viewpoints and see how each of them affect the overall throughput \footnote{and this
categorization is by no means exhaustive. We are naming only a handful.}.

\subsection{Consensus and Block Authoring}

As mentioned, the consensus protocol provides the means of ensuring that all nodes have a persistent
view of the state, and it can heavily contribute to the throughput of the system. As an example, two
common consensus protocols are Proof of \textbf{Work} and Proof of \textbf{Stake}. They use the
computation power (\textit{work}) and an amount of bonded tokens (\textit{stake}), respectively, as
their guarantees that an entity has \textit{authority} to perform some operation, such as authoring
a block. It is important to note that each of these consensus protocols has an \textit{inherently}
different throughput characteristics \cite{meneghettiSurveyEfficientParallelization2019}. Proof of
work, as the names suggests, requires the author to prove their legitimacy by providing a proof that
they have solved a particular hashing puzzle. This is slow by nature, and wastes a lot of
computation power on each node that wants to produce blocks, which in turn has a negative impact on
the frequency of blocks, which directly impacts the transaction throughput. Speeding-up this metric
requires the network to agree on an easier puzzle, that can in turn make the system less secure
\cite{gervaisSecurityPerformanceProof2016}.

To the contrary, Proof of Stake does not need this this puzzle solving, which is beneficial in terms
of computation resources. Moreover, since the chance of any node being the author is determined by
their stake\footnote{Using some hypothetical election algorithm which is irrelevant to this work.},
a smaller block-time is not insecure in itself. Recently, we are seeing blockchains turning into
verifiable random functions \cite{dodisVerifiableRandomFunction2005} for block authoring, and deploy
a traditional byzantine fault tolerance voting scheme on top of it to ensure finality
\cite{buterinCasperFriendlyFinality2019, stewartPosterGRANDPAFinality2019}. This further
\textit{decouples block production and finality}, allowing production to proceed faster and and with
even less drag from the rest of the consensus system, namely finality.

All in all, one general approach towards increasing the throughput of a blockchain is to
\textit{re-think the consensus and block authoring mechanisms} that dictate when blocks are added to
the chain, specifically it which frequency. It is crucially important to note that any approach in
this domain falls somewhere in the spectrum of centralized-decentralized, where most often
approaches that are more centralized will be more capable of delivering better throughput, yet they
may not have some of the security and immutability guarantees of a blockchain. An example of this is
provided in table \ref{table:blockchain_types} and where private blockchains were named as being
always the fastest.

\subsubsection{Sharding}

An interesting consensus-related optimisation that is gaining a lot of relevance in recent years is
a technique, obtained from the databases field, called \textit{sharding}. Shards are slices of data
(in the database jargon) that are maintained on different nodes. In a blockchain system, shards
refer to sub-chains that are maintained by sub-networks of the whole system. In essence, instead of
keeping \textit{all} the nodes in the entire system synchronized at \textit{all} times, sharded
blockchains consist of \textit{multiple} smaller networks, each maintaining their own cannon chain.
Albeit, most of the time these sub-chains all have the same prefix and only differ in the most
recent blocks. At fixed intervals, sub-networks come to agreement and synchronize their shards with
one another. In some sense, sharding allows smaller sub-networks to progress
faster\cite{forestierBlockcliqueScalingBlockchains2019, al-bassamChainspaceShardedSmart2017,
shreyDiPETransFrameworkDistributed2019}.

\subsection{Chain Topology}

Another approach is changing the nature of the chain itself. A classic blockchain is theoretically
limited due to its shape: a chain has only one head, thus only one new block can be added at each
point in time. This property will bring extra security, and make the chain state easier to reason
about (i.e. there is only one cannon chain). A radical approach is to question this property and
allow different blocks (or individual transactions) to be created at the same time. Consequently,
this approach turns a blockchain from a literal \textit{chain of blocks} into a \textit{graph of
transactions} \cite{pervezComparativeAnalysisDAGBased2018}. Most often, such technologies are
referred to Directed Acyclic Graphs (DAG) solutions. A prominent example of this is the IOTA
project\cite{mIOTANextGenerationBlock2018}.

Allowing the chain to grow from different heads (i.e. seeing it as a graph) allows true parallelism
at the transaction layer, effectively increasing the throughput. Nonetheless, the security of such
approaches is still an active area of research and achieving decentralization with such loose
authoring constraints has proven to be challenging \cite{sompolinskySPECTREFastScalable2016}.

Altering chain topology will bring even more radical changes to the original idea of blockchain.
While being very promising for some fields such as massively large user applications (i.e. "Internet
of Things", micro-payments), we will not consider DAGs in this work. We choose to adhere to the
definitions provided in chapter \ref{chap:background} as our baseline of \textit{what a blockchain
is}.

\subsection{Deploying Concurrency over Transaction Processing}
\label{chap_approach:subsec:out_of_box_concurrency}

Finally, we can focus on the transaction processing view of the blockchain, and try and deploy
concurrency on top of it, leaving the other aspects such as consensus unchanged and, more
importantly, \textit{generic}. This is very important, as it allows our approach to be deployed on
many chains, since it is independent of any chain-specific detail. Any chain will eventually come to
a point where it must execute some transactions, be it in the form of a chain, or a DAG, with any
consensus. Thus, concurrency is a viable as long as notion of transactions and blocks exists.

Our work specifically focuses on this aspect of of blockchain systems and proposes a novel approach
to achieve concurrency within each block's execution, both in the authoring phase and in the
validation phase, thereby increasing the throughput.

\subsection{Summary} \label{chap_bg:subsec:summary_speedup}

\todo[inline]{I am not yet super happy with this section. I think I can write it in a more compact +
confident way}.

Indeed, it is questionable how much gain will concurrency bring to the overall throughput. To
counter this doubt, we argue that the aim of this work is to, foremost, \textit{improve} the
throughput, not aiming for a specific absolute rate. It is true that in a specific chain, the
network latency or consensus process might be the main bottleneck of the \textit{overall
throughput}, and concurrency cannot improve that by much. For example, most performance gains within
concurrency might be within milliseconds, while the block authoring delay is tens of seconds in some
chain. Nonetheless, this does not mean that there is no merit to it. At any of these layers, any
speedup is valuable and translates to more efficient use of the hardware.

Therefore, we argue that our method is \textit{valuable} and \textit{applicable}, regardless of
being the dominant factor or not. For some chains, it might be a dominant factor and drastically
improve the throughput, whilst in others it might not be and only allow the system to better use the
hardware that is provided to it\footnote{It is worth noting that having optimal hardware utilization
(to reduce costs) is an important factor in the blockchain industry, as many chains are ran by
people who are making profit out of running validators and miners.}.

Finally, by seeing these broad options, we can clarify our usage of the word "throughput". One might
notice that the first two options mentioned in this section (consensus, chain topology) can increase
the throughput at the \textit{block} level: More blocks can be added, thus more transaction
throughput. This is in contrast to what concurrency can do. The concurrency explained in
\ref{chap_approach:subsec:out_of_box_concurrency} is the matter of what happens \textit{within}
\textit{\textbf{a}} block. Henceforth, by throughput we mean throughput of transactions that are
being executed within a (\textit{single}) block. Similarly, by concurrent, we mean concurrent within
the transactions of a (\textit{single}) block, not concurrency within the blocks themselves.


\section{Concurrency Within Block Production and Validation} \label{chap_approach:sec:concurrency}

In this section, we will explain in detail how a concurrent blockchain will function. Most notably,
we define how a concurrent author and a concurrent validator differ from their sequential
counterparts. Note that these are the mandatory requirements that \textit{any} approach toward
concurrency in blockchains must respect. As the reader might expect based on previous explanations,
all of them boil down to one radical property: \textbf{determinism}.

To recap, the block author is the elected entity that proposes a new block consisting of
transactions. The block author must have already executed these transaction in some
protocol-specific order (e.g. sequential), and note the correct state root of the block in its
header. This block is then propagated over the network. All other nodes validate this block and if
they all come to the same state root, they append it to their local chain. An author that
successfully creates a block gets rewarded for their work by the system.

\subsection{Concurrent Author}

We will begin by the chronologically sensible way, block authoring. Before anything interesting can
happen in a blockchain, someone has to author and propose a new block. Else, no state transition
happens.

A concurrent author has access to a pool of transactions that have been received over the network,
most often via the gossip protocol. Form a consensus point of view, it is absolutely irrelevant to
ensure all nodes have the same transactions in their local pool. In other words, form a consensus
point of view, there is no consensus in the transaction pool layer. All that matters is that any
node, once chosen to be the author\footnote{The decision of how the author is elected is within the
details of the consensus protocol.}, has a pool from which it can choose transactions. Then, the
author has a \textit{limited} amount of time to prepare the block and propagate it.

A number of ambiguities arise here. We will dismiss them all to be able to only focus on the
concurrency aspect.

\begin{itemize}
	\item Typically, the author need some way to \textit{prefer} a subset of the transactions pool,
	as most often all of it cannot be fit into the block. For this work, we leave this detail
	generic and assume that each author has first filtered out its pool into a new pool of
	transactions that she prefers to include in the block. In reality, a common strategy here is to
	prioritize the transactions that will pay off the most fee, as this will benefit the block
	author.
	\item A block must have some chain-specific \textit{resource} limit. For simplicity,we assume
	that each block can fit a fixed \textit{maximum} number of transactions, but it is so high that
	the bottleneck is not the transaction limit itself, but rather the amount of \textit{time} that
	tha author has to prepare the block. Thus, bolstering the importance of high throughput. In
	reality, some blockchains have adopted a similar approach (cap on \textit{number} of
	transactions), or limit the size of the block (cap on encoded \textit{byte length} of the
	transaction). Complex chains that support arbitrary code execution will even go further and
	limit the \textit{computation} cost of the transactions, such as Ethereum's gas
	metering\cite{perezBrokenMetreAttacking2020}.
\end{itemize}

Having all these parameters fixed, we can then focus on the block building part, namely executing
each transaction and placing it in the block. A sequential author would simply execute all the
transactions one by one (in some order of preference) up until the time limit, and calculate the new
state root. These transactions are then structured as a block. Concatenated with a header that notes
the state root, the block is ready to be propagated. The created block is an \textit{ordered}
container for transactions, therefore it can be re-executed deterministically trivially by
validators, as long as everyone does it sequentially.

A concurrent author's goal is to execute these transactions in a concurrent way, hoping to fit
\textit{more} of them in the same limited \textit{time}, while still allowing the validators to come
to the same state root. This is challenging because most often concurrency is non-deterministic.
Therefore, the author is expected to piggy-back some auxillary information to its block that allows
validators to execute it deterministically. maintaining determinism is the first and foremost
criteria of the concurrent author.

Moreover, the secondary criteria is a net positive gain in throughput. The concurrent author prefers
to be able to execute more transactions within a fixed time frame that she has for authoring, for
she will then be rewarded with more transaction fees.

\subsection{Concurrent Validator}

A validator's role is simpler in both the sequential and concurrent fashion. Recall that the a block
is an ordered container for transactions. Then, the sequential validator has a trivial role:
re-execute the transactions sequentially and compare state roots. The concurrent validator however,
is likely to need to do more.

More specifically, the concurrent validator knows that a concurrent author must have executed all or
some of the transactions within the block concurrently. Therefore, conflicting transactions must
have \textbf{proceeded} one another in \textit{some} way. The goal of the concurrent validator is to
re-produce the \textit{same precedence} in an efficient manner and come to the same state root.

For example, Assume the author uses a simple Mutex lock to perform concurrency. In this case, some
transactions will inevitably have to wait for other transactions that accessed the same mutex
earlier. This implies an \textit{implicit} precedence between conflicting transactions. The author
will need to somehow transfer these precedence information to the validator, and the validator must
respect them in order to arrive at the same state root.

With this background, we will briefly survey existing approaches in the literature to achieve
concurrency, effectively seeing some a more practical perspective of the above description.

\section{Existing Approaches}

\todo[inline]{I am a bit tempted: Do I even need a related work section after all?}

In this section we will look at some of the already existing approaches toward concurrency. While
doing so, we denote their deficiencies and build upon them to introduce our approach.

Every tool that we named for concurrency in chapter \ref{chap:background} can essentially be used in
blockchains as well, yet each will have a specific toll on the system in order to be feasible. We
will begin by arguably the simplest, locking.

A locking approach would divide the transactions into multiple threads\footnote{a 1:1 relation
between threads and transactions is also possible, given that the programming language supports
green threads.}. Each transaction within the thread, when attempting to access any key in the
state\footnote{Recall that the state is a key value database.}, has to acquire a lock for it. Once
acquired, the transaction can access the key. As mentioned this process is not deterministic.
Therefore, the runtime need to keep track of which locks were requested by which thread, and the
\textit{order} in which they were granted. This information is, in essence, builds a dependency
graph. This dependency graph need to be sent to the validators as well. The validators parse the
dependency graph and based upon that spawn the required number of threads, and distribute the
transactions within the threads. \cite{dickersonSmartLocksAddingConcurrency2017} is among the
earliest works on concurrency within blockchain and adopts such an approach.

The details of generating the dependency graph with minimum size, encoding it in the block in an
efficient way and parsing it in the validator is being highly simplified here. These steps are
critical, as they are the main overheads of this approach. The size of this graph need to be small,
as it needs to be added to the block and increases the network overhead. Moreover, the overhead of
this extra processing must be worthwhile for the author, as otherwise it would be in contrast to the
whole objective of deploying concurrency. There are some works that only focus on the "dependency
graph generating and processing" aspect of the process. They assume some means exists through which
the read and write set of each transaction can be computed (i.e. by monitoring the lock requests
that each thread sends at runtime). On top of this, the provide efficient ways to build the
dependency graph, and use it at the validator's end \cite{EnablingConcurrencySmart2018}.

The next step of this progression is to utilize transactional memory. The line of research exactly
follows the same pattern. More recent works use software transactional memory to reduce the waiting
time and conflict rates. Similar to the locking approach, the runtime needs to keep track of the
dependencies and build a graph that encode this information. Different flavours of STMs are used and
compared in this line of research, such as Read-Write STMs, Single-Version Object-based STMs, and
Multi-Version Object-based STMs
\cite{anjanaSTMEfficientConcurrent2019,anjanaSTMEfficientFramework2019}. Nonetheless, the underlying
procedure stays the same: Some means of concurrency control to handle conflicts, track dependency
and use it to encode precedence, then re-crete the same precedence in the validator.

Next, we name an out-of-the box work that take a rather slightly different path. Many studies in the
blockchain literature use datasets from database industry as their reference. Such datasets might
have unrealistically high rates of contention. \cite{saraphYOLOEmpiricalStudy2019} is an empirical
study that tries to determine the conflict rates within Ethereum transactions, a \textit{live}, and
arguably among the well adopted, network. And, while doing so, it demonstrate a different, wait-free
approach. In the concurrent simulator of this work, all transaction are executed in parallel with
the assumption that they will not conflict. If a conflict happens and a transaction aborts, it is
discarded, and re-executed again at later phase sequentially. This essentially clusters transactions
into two groups: concurrent and sequential. All of the concurrent transactions are guaranteed not to
conflict. The sequential transaction do not matter as they are executed sequentially. Regardless of
their findings about the conflict rates in different periods of time in Ethereum, they also report
speedups in \textit{some} cases, not being too shy of the speedup amounts reported by
\cite{dickersonSmartLocksAddingConcurrency2017}, which uses locking.

This is an inspiring finding, implying that perhaps complicated concurrency control might not be
needed after all for many of the transactions in \textit{some} periods of time, based on the
contention of the transactions. In some sense, this work adopts a technique that we coin as
\textbf{concurrency avoidance} instead of concurrency control. As a consequence, the system need not
to deal with conflicts in any way, because they are rejected and dealt with separately in the
sequential phase.

Finally, we note that there has also been some work on static analysis in the field of blockchains,
yet all of those that we have found require fundamental changes to the programmable language of the
target chain. For example, \cite{bartolettiStaticTrueConcurrent2019} provides an extension to the
Ethereum's smart contract language, Solidity, that allow it to be executed in a truly concurrent
manner. Similarly, RChain is an industrial example of a chain that has a programming model that is
fundamentally concurrent\cite{darrylRCast21Currency2019}, namely
pi-calculus\cite{turnerPolymorphicPiCalculusTheory1996}. Such approaches are also inspiring, yet we
prefer devising an approach which does not need to alter such fundamental assumptions about the
programming model, in favour of easy adoption and outreach.



\begin{remark}
	Most of the mentioned references name their work as appraoches toward concurrency for
	\textbf{smart contracts}. At this point, it would be helpful to clarify that. The details of
	smart contract chains as well beyond the scope of this work. But, it is worth noting that a
	smart contract chain is a fixed chain that a fixed state transition logic that, and a part of
	that logic is to store codes (smart contracts) and execute them upon being dispatched. Moreover,
	since Ethereum is the prominent smart contract chain, all of these works present themselves with
	simulators that can hypothetically be implemented in the Ethereum node. To the contrary, we
	won't limit ourself to smart contracts or any specific chain in this work and build upon the
	basis that the future of blockchains will not be a \textit{single} chain (chain maximalism), but
	rather an abundance of domain specific chains interoperating with one another. To achieve this,
	one needs to think in the context of a framework for building blockchains, not a particular
	blockchain per se. Thus, we argue that our approach is applicable to any chain with independent
	runtimes, not any particular smart contract being executed in the runtime of another chain.
\end{remark}

All of the works mentioned above report positive speedups. Yet, we argue differently and point out
some overheads associated with them that can be improved, or challenges that can be avoided
altogether. Both locking and transactional memory will result in a sizeable overhead while
authoring. This is mostly hidden in some sort of runtime overhead, for example the need to keep
track of locking order, and consequently parsing it into a dependency graph. Moreover, this will
inevitably increase the block size. Finally, the validator also needs to tolerate the overhead of
parsing the dependency graph and making decisions based upon it.

These are all overheads compared to the basic sequential model. In essence, we express skepticism
toward these complex runtime machinery to deal with conflicts, and record precedence. On the other
hand the pure \textit{concurrency avoidance} model is likely to fail under any workload with some
non-negligible degree of contention, because it basically falls back to the sequential model where
most transactions are aborted and moved to the sequential model. Results from
\cite{saraphYOLOEmpiricalStudy2019} show the same trend.

In our approach, introduced in the next section, we try to minimize these overheads by finding a new
balance between "concurrency control" against "concurrency avoidance" model. Moreover, we apply the
same analogy between "runtime" against "static". While \textit{some} runtime apparatus is needed to
orchestrate the execution and prevent chaos, tracking all dependencies is likely to be too much.
Similarly, while a purely static approach toward concurrency is a radical change to common
programming models, \textit{some} static hints could nudge the runtime by providing useful
information.

----

\section{Our Approach: Almost Lock-Free, Concurrency Delegation and Pseudo Static}
\label{chap_desgin:sec:design}

\todo[inline]{Basically https://www.notion.so/Architecture-V1-cfe987d4dd4345569b07649dcf00a85c end
of the chapter: say that we have answered the RQ1.}

\todo[inline]{I like to coin stuff. Maybe I can come up with some fancy middleground name for tx forwarding? So far I call them concurrency Avoidance and Concurrency Control. Maybe we can call this one delegatory concurrency? Fuck yeah that sounds good.}

In this section, we will describe our approach toward concurrency in a blockchain runtime, both in
the authoring phase and in validation. This approach is based on three key pivotal ideas, explained
in the next section.

\subsection{Key Ideas}

The key ideas of our approache can be enumerated as follows:

\subsubsection{Almost Lock-Free: "Taintable" State}\label{chapt_approach:subsubsec:taintable_state}

We have already seen that locking is a common primitive to achieve shared state concurrency. In our
approach, we relax this primitive such that any access to a shared state by a thread \textbf{does
not incur long waiting times}, but instead, it might simply fail. To do so, we link each key in the
state database with a taint value. If a key has never been accessed before, it is untainted. Once it
is accessed by any thread (regardless of the type of operation), it is tainted by the identifier
of that accessing thread. Henceforth, any access to this key simply fails, returning the identifier of the
original tainter (aka. \textit{owner}) of the key. As we will see in the implementation details,
this approach is \textit{almost} wait free, meaning that threads will almost always proceed
immediately with any state operation. Indeed, a thread can always freely access keys that it has
already tainted before.

\subsubsection{Concurrency Delegation}

If a thread, in the process of executing a transaction, tries to access a state key which has
already been tainted before, then it forwards this transaction to the owner of that key. By doing
so, a thread basically \textit{delegates} the task of executing a transaction concurrently to
another thread, namely because it cannot meet the state-access requirements of the transactions
itself. This is the middle ground between \textit{concurrency avoidance} and \textit{concurrency
control}, which we have coined \textit{concurrency delegation}.

Compared to concurrency control, threads do not try to resolve contention in any sophisticated way.
Instead, they will simply delegate (aka. \textit{forward}) the transactions to whoever they think
\textit{might} be able to execute it successfully. There is no waiting involved, and no record being
kept.

On the other hand, comparing this with concurrency avoidance model, just because a transaction 's
state operation has failed, it does not mean that it cannot be executed in any concurrent way.
Instead, it might be executable by another thread who is known to be the owner of the problematic
state key. Therefore, instead of immediately being discarded (and potentially executed sequentially
at the end), each transaction is given a second chance to succeed in a concurrent fashion.

Finally, if a transaction is already forwarded and still cannot be executed, (only) then it is
forwarded to a sequential queue to be dealt with later. We name such transactions \textbf{orphan}.

\subsubsection{Pseudo-Static}

We predict that concurrency delegations works well when threads need to seldom forward transactions
to one another. In other words, the transactions need to be initially distributed between the
threads in some constructive and effective way to minimize forwarding. This is where we turn to
pseudo-static heuristics as a form of a hint. We use the term pseudo-static because because we
do not mean information that is necessarily available at compile time, but rather are known
\textit{before} a particular transactions is executed. In other words, they are \textit{static with
respect to the lifetime of a transaction} and can be inferred without the need to \textit{execute}
a transaction, but rather by \textit{inspecting} it.

Each of these ideas, along with the utmost details of their ramifications and delicacies are
explained in the rest of this section.

\subsection{Baseline Algorithm}

\begin{remark}
	For ease of understanding, we assume a system with 1 main (master) thread and 4 worker threads
	in this section. Needless to say, all of the details are applicable to systems with more or less
	degrees of parallelism.
\end{remark}

We will first look into the baseline algorithm, which is essentially the concurrency delegation
model without any static heuristics. We assume a single main thread\todo{It probably reads easier if
I just replace "main thread" with "Master" all over the place.}, and $4$ worker threads, each having
the ability send messages over channels to one another. The main thread has access to a potentially
infinitely large pool of (\textit{ordered}, ready to execute) transactions. Moreover, it has an
(initially) empty pool of orphan transactions, to which worker threads can forward transaction.
Moreover, each worker has a local queue, to which the main thread and other workers can send
transactions. The main thread and all workers share the reference to the same state database, $S$,
which follows the logic of a taintable state as explained in
\ref{chapt_approach:subsubsec:taintable_state}.

\begin{remark}
	As we will see, it is crucial to remember that both the transaction pool and the block are
	\textbf{ordered} containers for transactions. In other words, they both act like an ordered
	list/array/vector of transactions. The order of transaction in the pool will end up being used
	to order the transaction in the final block as well. \todo{which reminds we, maybe I should call
	this transaction queue instead of pool then, or clarify this somewhere.}
\end{remark}

The main thread's (simplified) execution logic during \textbf{authoring} is as follows:

\begin{enumerate}
	\item \textbf{Distribution phase}: The main thread starts distributing transactions between
	threads by some arbitrary function $F$. In essence, $F$ is a $fn(transaction) \rightarrow identifier$,
	meaning that for each transaction, it outputs one thread identifier. Once the distribution is
	done, each transaction in the pool is \textit{tagged} by the identifier if one worker thread.
	The distribution phase ends with the main thread sending each transaction to its corresponding
	worker thread's local queue.

	\item \textbf{Collection phase}: The main thread will then wait for reports from all workers,
	indicting that they are done with executing all of the transactions that they have received
	earlier. During this phase, threads might forward transactions to one another, and might forward
	transactions back to the main thread, if deemed to be orphan, exactly as explained in the
	concurrency delegation model. Both of these events are reported to the main thread and the tag
	of each transaction might change in the initial pool. Once termination is detected by the main
	thread, a message is sent to all workers to shut them down.

	\item \textbf{Orphan phase}: Once all worker threads are done, the main thread will execute any
	transactions that it may have received in its orphan pool. At this point, the main thread is
	sure that there are no other active threads in the system, thus accessing $S$ without worrying
	about any of the taint values is safe. The transactions are executed sequentially on top of $S$,
	and their tag is changed to a special identifier for orphan transactions.
\end{enumerate}

Then, the main thread is ready to finalize the block. By this point, each transaction is either
tagged to be orphan, or with the identifier of one of the $4$ worker threads. In essence, we have
clustered transactions into $4 + 1$ \textit{ordered} groups. The validator who receives this block
will respect this clustering and execute transactions with the same tag in the same thread. In the
first $4$ groups, the transactions within a group might conflict with one another (e.g. attempt to
write to the same key), but they are ordered and are known the be executed by the same thread, thus
safe. The transactions in the last group, namely the orphan group, are executed sequentially and in
isolation, thus safe.

In the same time frame, the worker thread will do as such (with slight simplifications):

\begin{enumerate}
	\item \textbf{Depleting local queue}: Having received a number of transactions from the main
	thread (after the "\textit{Distribution phase}"), each worker will then try to deplete its local
	queue. For each transaction $Tx$, the logic is as follows:

	If $Tx$ is executed successfully, nothing is done or reported. This is because the main thread
	is already \textit{assuming} that $Tx$ \textit{will be executed} by the current worker thread,
	thus nothing need to be reported. If $Tx$ fails due to a taint error, it is forward to the owner
	thread that of the state keys that caused the failure. Note that at this point we know that $Tx$
	has not been forwarded before, because it is being retrieved from the initial local queue.

	At the end of this phase, the worker will send an overall report to the main thread, noting how
	many of the transactions it could execute successfully, and how many ended up being forwarded.
	This data is then used at the main thread to detect termination.

	\item \textbf{Termination phase}:Once done with their local queue, the threads will listen for
	two messages, namely \textit{termination} or \textit{forwarded} transactions from other threads.
	Termination is the message from the main thread to shut down the worker. Forwarded transactions
	are those that another worker thread is \textit{delegating} to the current thread, namely
	because of a taint error. The transaction is then executed locally and if it is successful, the
	result is reported to the main thread. If the execution fails again due to a taint error, then
	it is forwarded to the main thread as an orphan.

	Note that in this case, reporting is vital, because a thread is ending up executing a
	transaction and the main thread is not aware of this, because it is not the same tag assigned in
	the "Distribution phase" of the main thread. This is also needed for the termination detection
	of this phase.
\end{enumerate}


A number of noteworthy remarks exists on the baseline algorithm:

\textbf{Termination Detection}. We intentionally did not describe how the main threads detects the
termination of the collection phase, because in doing so we needed further information from the
worker thread's logic. Recall that the main thread knows how many transactions it has initially
distributed between all the workers. Moreover, from the reports sent by thw worker at the end of
"Depleting local queue" phase, it knows how many of them executed \textit{locally} in their
\textit{designated thread}, and how many of them ended up being \textit{forwarded}. Also, recall
that each forwarded transaction, upon being executed successfully, is reported to the main thread.
Similarly, each forwarded transaction that fails is also reported to the main thread (by being
forwarded to the orphan queue residing in the main thread). Thus, the main thread can safely assert
that: Termination is achieved once the sum of "all locally executed transactions at workers",
"forwarded and successfully executed transactions" and "orphan transactions" is equal to the initial
count of transactions in the pool. At this point, the termination message is created and broadcasted
to all workers.

\begin{lemma}
	\textbf{Termination of the main thread's collection phase. }

	Assume $N$ initial transaction. Each worker thread, upon finishing the "deplete local queue"
	phase reports back $\{ l_{1}, l_{2}, l_{3}, l_{4} \}$, indicating the number of transactions
	that a worker executed locally. Given $F$ being the number of reports of transactions being
	forwarded, and $O$ being the size of the orphan pool, termination is achieved once:

	\vspace{0.3cm}
	$N == \sum_{t = 1}^{4} l_{t} + F + O$
\end{lemma}

\textbf{Maintaining Order}: Aside from termination detection, it is also vital for determinism that
the main thread takes action upon the report of a transaction being forwarded. This is because
\textbf{once the tag of a transaction changes, it is likely that its order must also change within
the pool}. For example, if a transaction, initially assigned to $T_{1}$ is known to be forwarded and
executed by $T_{2}$, it is important to re-order it in the initial transaction pool such that it is
placed after all the transactions initially assigned to $T_{2}$. This is because, in reality,
$T_{2}$ first executed all of its designated transactions and then executes any forwarded
transactions. Recall that the pool is an ordered container for transaction and its order will
eventually end up building the order of transaction in the block.

\textbf{Orphan Transactions}. Recall that an orphan transaction is a transaction has already been
forwarded and still fails to execute, at its current host thread, due to taint error. We now
represent this from a different perspective. In our concurrency delegation scheme, threads race to
access state keys and upon successful access, they taint them. Any transaction has a number of state
keys that it needs to access in order to be processed. \textbf{An orphan transaction is one that has
state keys being tainted by at least two \textit{different} threads}. For example, assume a
transaction needs to access keys $K_{1}$ and $K_{2}$. Assume thread $T_{0}$ is executing this
transaction. If $K_{1}$ is already tainted by a $T_{1}$ and $K_{2}$ by a $T_{2}$, then this
transaction will inevitably end up being in the orphan queue. The transaction is first forwarded
from $T_{0}$ to $T_{1}$, where it can successfully access $K_{1}$, but still fails at accessing
$K_{2}$ and thus orphaned\footnote{An interesting optimization can be applied on top of this logic,
which is explained further in TODO.}.

\textbf{Minimal Overhead}. As we have seen, our approach incurs minimal overhead to the block. In
fact, the only additional data needed is one identifier that need to be attached to each
transaction, indicating which thread must execute it (and the special case thereof, orphan
transaction), essentially the \textit{tag}. This can be as small is a single byte per transaction,
which is not much. Note that transactions within the block still maintain partial order: The
transactions of a particular tag are sorted within their tag. Only the relative order of
transactions from different tags is lost, which is not significant, because they are guaranteed to
not conflict.

We can see now consider \textbf{validation} as well. As expected, due to the minimal overhead and
the simplicity of the baseline algorithm, the validation logic is fairly simple. Each block is
received with all of its transactions having a tag. The ones tagged to be orphan are set aside for
later execution. Then, one worker thread is spawned per tag and transactions are assigned to threads
based on their tags, and in the same relative order. The workers can then execute concurrently,
without the need for any concurrency control, we because they effectively know that all contentious
transactions already have the same tags, and thus are ordered \textit{sequentially} \textit{within
that thread}. In essence, the validation is fully \textit{parallelizable} and does not need any
synchronization. Once all threads are finished, the orphan transactions are executed sequentially
and validation comes to an end.

Finally, we must address the most important criteria, \textbf{Determinism}. We provide the proof by
showing that the validation and authoring both have the exact same execution environment, and the
threads are executed in the exact same order in both phases. In more detail, both the author and the
validator will spawn the same number of threads, and all of the transaction executed by any thread
during authoring are re-executed by a single thread in the validation phase as well.

Recall again that the pool has a particular order (based on an arbitrary preference) and the same
order will be imposed on the authored block as well.

First, consider all transactions that are executed in their initially designated thread, based on
the aforementioned distributer function $F$. For all of the transaction of a particular tag, they
are placed in the pool in the same partial order as they were executed. Therefore, they are also
placed in the block with the same partial order. Upon validation, all of the transactions of this
particular tag are assigned to a single thread, that executed them sequentially in the same order,
thus deterministic. The same logic can trivially be applied to all order tags as well, essentially
one per worker thread.

Next, consider the transactions that ended up being forwarded. These transactions ended up being
executed after the designated transactions of their final host thread. The main thread, responsible
for building the final block, have to note this change and ensure that this partial order is
maintained in the final block. This is achieved by re-ordering the forwarded transactions to the end
of the queue. Given this, we again have the guarantee that the transactions are placed in the block
in the same partial order as they were executed, therefore the same logic as with in the previous
paragraph can be used to derive d

\todo[inline]{I feel like this relative order can be explained a bit better.}

// TODO: now I am not sure if 100% respect the order in the code. The workers will each def. execute
their assigned transactions. Once done, they will execute a forwarded transaction and immediately
report.


\subsection{Applying Static Heuristics}

\subsection{System Design}


