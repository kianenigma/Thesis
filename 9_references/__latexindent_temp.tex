 @article{Al-Bassam_Sonnino_Bano_Hrycyszyn_Danezis_2017, title={Chainspace: A Sharded Smart Contracts Platform}, url={http://arxiv.org/abs/1708.03778}, abstractNote={Chainspace is a decentralized infrastructure, known as a distributed ledger, that supports user defined smart contracts and executes user-supplied transactions on their objects. The correct execution of smart contract transactions is verifiable by all. The system is scalable, by sharding state and the execution of transactions, and using S-BAC, a distributed commit protocol, to guarantee consistency. Chainspace is secure against subsets of nodes trying to compromise its integrity or availability properties through Byzantine Fault Tolerance (BFT), and extremely highauditability, non-repudiation and ‘blockchain’ techniques. Even when BFT fails, auditing mechanisms are in place to trace malicious participants. We present the design, rationale, and details of Chainspace; we argue through evaluating an implementation of the system about its scaling and other features; we illustrate a number of privacy-friendly smart contracts for smart metering, polling and banking and measure their performance.}, note={arXiv: 1708.03778}, journal={arXiv:1708.03778 [cs]}, author={Al-Bassam, Mustafa and Sonnino, Alberto and Bano, Shehar and Hrycyszyn, Dave and Danezis, George}, year={2017}, month={Aug} }
 @book{Antonopoulos_Wood_2019, place={Sebastopol, CA}, edition={First edition}, title={Mastering Ethereum: building smart contracts and DApps}, ISBN={978-1-4919-7194-9}, abstractNote={Ethereum represents the gateway to a worldwide, decentralized computing paradigm. This platform enables you to run decentralized applications (DApps) and smart contracts that have no central points of failure or control, integrate with a payment network, and operate on an open blockchain. With this practical guide, Andreas M. Antonopoulos and Gavin Wood provide everything you need to know about building smart contracts and DApps on Ethereum and other virtual-machine blockchains. Discover why IBM, Microsoft, NASDAQ, and hundreds of other organizations are experimenting with Ethereum. This essential guide shows you how to develop the skills necessary to be an innovator in this growing and exciting new industry}, publisher={O’Reilly}, author={Antonopoulos, Andreas M. and Wood, Gavin}, year={2019} }
 @article{Bano_Sonnino_Al-Bassam_Azouvi_McCorry_Meiklejohn_Danezis_2017, title={Consensus in the Age of Blockchains}, url={http://arxiv.org/abs/1711.03936}, abstractNote={The blockchain initially gained traction in 2008 as the technology underlying bitcoin, but now has been employed in a diverse range of applications and created a global market worth over $150B as of 2017. What distinguishes blockchains from traditional distributed databases is the ability to operate in a decentralized setting without relying on a trusted third party. As such their core technical component is consensus: how to reach agreement among a group of nodes. This has been extensively studied already in the distributed systems community for closed systems, but its application to open blockchains has revitalized the field and led to a plethora of new designs. The inherent complexity of consensus protocols and their rapid and dramatic evolution makes it hard to contextualize the design landscape. We address this challenge by conducting a systematic and comprehensive study of blockchain consensus protocols. After first discussing key themes in classical consensus protocols, we describe: first protocols based on proof-of-work (PoW), second proof-of-X (PoX) protocols that replace PoW with more energy-efficient alternatives, and third hybrid protocols that are compositions or variations of classical consensus protocols. We develop a framework to evaluate their performance, security and design properties, and use it to systematize key themes in the protocol categories described above. This evaluation leads us to identify research gaps and challenges for the community to consider in future research endeavours.}, note={arXiv: 1711.03936}, journal={arXiv:1711.03936 [cs]}, author={Bano, Shehar and Sonnino, Alberto and Al-Bassam, Mustafa and Azouvi, Sarah and McCorry, Patrick and Meiklejohn, Sarah and Danezis, George}, year={2017}, month={Nov} }
 @article{Baran_1964, title={On Distributed Communications Networks}, volume={12}, ISSN={1558-2647}, DOI={10.1109/TCOM.1964.1088883}, abstractNote={This paper briefly reviews the distributed communication network concept in which each station is connected to all adjacent stations rather than to a few switching points, as in a centralized system. The payoff for a distributed configuration in terms of survivability in the cases of enemy attack directed against nodes, links or combinations of nodes and links is demonstrated. A comparison is made between diversity of assignment and perfect switching in distributed networks, and the feasibility of using low-cost unreliable communication links, even links so unreliable as to be unusable in present type networks, to form highly reliable networks is discussed. The requirements for a future all-digital data distributed network which provides common user service for a wide range of users having different requirements is considered. The use of a standard format message block permits building relatively simple switching mechanisms using an adaptive store-and-forward routing policy to handle all forms of digital data including digital voice. This network rapidly responds to changes in the network status. Recent history of measured network traffic is used to modify path selection. Simulation results are shown to indicate that highly efficient routing can be performed by local control without the necessity for any central, and therefore vulnerable, control point.}, number={1}, journal={IEEE Transactions on Communications Systems}, author={Baran, P.}, year={1964}, month={Mar}, pages={1–9} }
 @book{BASHIR_2018, place={Place of publication not identified}, title={MASTERING BLOCKCHAIN: distributed ledger technology, decentralization, and smart contracts explained, 2nd edition;distributed ledger.}, ISBN={978-1-78883-867-2}, url={http://search.ebscohost.com/login.aspx?direct=true&scope=site&db=nlebk&db=nlabk&AN=1789486}, publisher={PACKT Publishing}, author={BASHIR, IMRAN}, year={2018} }
 @inproceedings{Bellare_Canetti_Krawczyk_1996, place={Berlin, Heidelberg}, series={Lecture Notes in Computer Science}, title={Keying Hash Functions for Message Authentication}, ISBN={978-3-540-68697-2}, DOI={10.1007/3-540-68697-5_1}, abstractNote={The use of cryptographic hash functions like MD5 or SHA-1 for message authentication has become a standard approach in many applications, particularly Internet security protocols. Though very easy to implement, these mechanisms are usually based on ad hoc techniques that lack a sound security analysis.We present new, simple, and practical constructions of message authentication schemes based on a cryptographic hash function. Our schemes, NMAC and HMAC, are proven to be secure as long as the underlying hash function has some reasonable cryptographic strengths. Moreover we show, in a quantitative way, that the schemes retain almost all the security of the underlying hash function. The performance of our schemes is essentially that of the underlying hash function. Moreover they use the hash function (or its compression function) as a black box, so that widely available library code or hardware can be used to implement them in a simple way, and replaceability of the underlying hash function is easily supported.}, booktitle={Advances in Cryptology — CRYPTO ’96}, publisher={Springer}, author={Bellare, Mihir and Canetti, Ran and Krawczyk, Hugo}, editor={Koblitz, Neal}, year={1996}, pages={1–15}, collection={Lecture Notes in Computer Science} }
 @book{Brumley_Tuveri_2011, title={Remote Timing Attacks are Still Practical}, url={https://eprint.iacr.org/2011/232}, abstractNote={For over two decades, timing attacks have been an active area of research within applied cryptography. These attacks exploit cryptosystem or protocol implementations that do not run in constant time. When implementing an elliptic curve cryptosystem with a goal to provide side-channel resistance, the scalar multiplication routine is a critical component. In such instances, one attractive method often suggested in the literature is Montgomery’s ladder that performs a fixed sequence of curve and field operations. This paper describes a timing attack vulnerability in OpenSSL’s ladder implementation for curves over binary fields. We use this vulnerability to steal the private key of a TLS server where the server authenticates with ECDSA signatures. Using the timing of the exchanged messages, the messages themselves, and the signatures, we mount a lattice attack that recovers the private key. Finally, we describe and implement an effective countermeasure.}, number={232}, author={Brumley, Billy Bob and Tuveri, Nicola}, year={2011} }
 @inproceedings{Carlucci_De Cicco_Mascolo_2015, place={Salamanca, Spain}, series={SAC ’15}, title={HTTP over UDP: an experimental investigation of QUIC}, ISBN={978-1-4503-3196-8}, url={https://doi.org/10.1145/2695664.2695706}, DOI={10.1145/2695664.2695706}, abstractNote={This paper investigates “Quick UDP Internet Connections” (QUIC), which was proposed by Google in 2012 as a reliable protocol on top of UDP in order to reduce Web Page retrieval time. We first check, through experiments, if QUIC can be safely deployed in the Internet and then we evaluate the Web page load time in comparison with SPDY and HTTP. We have found that QUIC reduces the overall page retrieval time with respect to HTTP in case of a channel without induced random losses and outperforms SPDY in the case of a lossy channel. The FEC module, when enabled, worsens the performance of QUIC.}, booktitle={Proceedings of the 30th Annual ACM Symposium on Applied Computing}, publisher={Association for Computing Machinery}, author={Carlucci, Gaetano and De Cicco, Luca and Mascolo, Saverio}, year={2015}, month={Apr}, pages={609–614}, collection={SAC ’15} }
 @inproceedings{Chaum_Fiat_Naor_1990, place={New York, NY}, series={Lecture Notes in Computer Science}, title={Untraceable Electronic Cash}, ISBN={978-0-387-34799-8}, DOI={10.1007/0-387-34799-2_25}, abstractNote={The use of credit cards today is an act of faith on the p a t of all concerned. Each party is vulnerable to fraud by the others, and the cardholder in particular has no protection against surveillance.}, booktitle={Advances in Cryptology — CRYPTO’ 88}, publisher={Springer}, author={Chaum, David and Fiat, Amos and Naor, Moni}, editor={Goldwasser, Shafi}, year={1990}, pages={319–327}, collection={Lecture Notes in Computer Science} }
 @inproceedings{Demers_Greene_Hauser_Irish_Larson_Shenker_Sturgis_Swinehart_Terry_1987, place={Vancouver, British Columbia, Canada}, series={PODC ’87}, title={Epidemic algorithms for replicated database maintenance}, ISBN={978-0-89791-239-6}, url={https://doi.org/10.1145/41840.41841}, DOI={10.1145/41840.41841}, booktitle={Proceedings of the sixth annual ACM Symposium on Principles of distributed computing}, publisher={Association for Computing Machinery}, author={Demers, Alan and Greene, Dan and Hauser, Carl and Irish, Wes and Larson, John and Shenker, Scott and Sturgis, Howard and Swinehart, Dan and Terry, Doug}, year={1987}, month={Dec}, pages={1–12}, collection={PODC ’87} }
 @article{Diffie_Hellman_1976, title={New directions in cryptography}, volume={22}, ISSN={1557-9654}, DOI={10.1109/TIT.1976.1055638}, abstractNote={Two kinds of contemporary developments in cryptography are examined. Widening applications of teleprocessing have given rise to a need for new types of cryptographic systems, which minimize the need for secure key distribution channels and supply the equivalent of a written signature. This paper suggests ways to solve these currently open problems. It also discusses how the theories of communication and computation are beginning to provide the tools to solve cryptographic problems of long standing.}, number={6}, journal={IEEE Transactions on Information Theory}, author={Diffie, W. and Hellman, M.}, year={1976}, month={Nov}, pages={644–654} }
 @inproceedings{Dodis_Yampolskiy_2005, place={Berlin, Heidelberg}, series={Lecture Notes in Computer Science}, title={A Verifiable Random Function with Short Proofs and Keys}, ISBN={978-3-540-30580-4}, DOI={10.1007/978-3-540-30580-4_28}, abstractNote={We give a simple and efficient construction of a verifiable random function (VRF) on bilinear groups. Our construction is direct. In contrast to prior VRF constructions [14,15], it avoids using an inefficient Goldreich-Levin transformation, thereby saving several factors in security. Our proofs of security are based on a decisional bilinear Diffie-Hellman inversion assumption, which seems reasonable given current state of knowledge. For small message spaces, our VRF’s proofs and keys have constant size. By utilizing a collision-resistant hash function, our VRF can also be used with arbitrary message spaces. We show that our scheme can be instantiated with an elliptic group of very reasonable size. Furthermore, it can be made distributed and proactive.}, booktitle={Public Key Cryptography - PKC 2005}, publisher={Springer}, author={Dodis, Yevgeniy and Yampolskiy, Aleksandr}, editor={Vaudenay, Serge}, year={2005}, pages={416–431}, collection={Lecture Notes in Computer Science} }
 @inproceedings{Dwork_Naor_1993, place={Berlin, Heidelberg}, series={Lecture Notes in Computer Science}, title={Pricing via Processing or Combatting Junk Mail}, ISBN={978-3-540-48071-6}, DOI={10.1007/3-540-48071-4_10}, abstractNote={We present a computational technique for combatting junk mail in particular and controlling access to a shared resource in general. The main idea is to require a user to compute a moderately hard, but not intractable, function in order to gain access to the resource, thus preventing frivolous use. To this end we suggest several pricing functions, based on, respectively, extracting square roots modulo a prime, the Fiat-Shamir signature scheme, and the Ong-Schnorr-Shamir (cracked) signature scheme.}, booktitle={Advances in Cryptology — CRYPTO’ 92}, publisher={Springer}, author={Dwork, Cynthia and Naor, Moni}, editor={Brickell, Ernest F.}, year={1993}, pages={139–147}, collection={Lecture Notes in Computer Science} }
 @article{Forestier_Vodenicarevic_Laversanne-Finot_2019, title={Blockclique: scaling blockchains through transaction sharding in a multithreaded block graph}, url={http://arxiv.org/abs/1803.09029}, abstractNote={Decentralized crypto-currencies based on the blockchain architecture are unable to scale to thousands of transactions per second. We define an architecture, called the blockclique, that addresses this limitation by sharding transactions in a block graph with multiple threads. A block in a given thread only contains transactions with input addresses assigned to this thread, and references one previous block of each thread as parents. When combined with a Proof-of-Stake node selection mechanism, the blockclique architecture reaches 10,000 transactions per second with a transaction time of less than a minute in our simulations, while being robust against known and projected attacks.}, note={arXiv: 1803.09029}, journal={arXiv:1803.09029 [cs]}, author={Forestier, Sébastien and Vodenicarevic, Damir and Laversanne-Finot, Adrien}, year={2019}, month={Sep} }
 @inproceedings{Gervais_Karame_Wüst_Glykantzis_Ritzdorf_Capkun_2016, place={Vienna, Austria}, series={CCS ’16}, title={On the Security and Performance of Proof of Work Blockchains}, ISBN={978-1-4503-4139-4}, url={https://doi.org/10.1145/2976749.2978341}, DOI={10.1145/2976749.2978341}, abstractNote={Proof of Work (PoW) powered blockchains currently account for more than 90% of the total market capitalization of existing digital cryptocurrencies. Although the security provisions of Bitcoin have been thoroughly analysed, the security guarantees of variant (forked) PoW blockchains (which were instantiated with different parameters) have not received much attention in the literature. This opens the question whether existing security analysis of Bitcoin’s PoW applies to other implementations which have been instantiated with different consensus and/or network parameters. In this paper, we introduce a novel quantitative framework to analyse the security and performance implications of various consensus and network parameters of PoW blockchains. Based on our framework, we devise optimal adversarial strategies for double-spending and selfish mining while taking into account real world constraints such as network propagation, different block sizes, block generation intervals, information propagation mechanism, and the impact of eclipse attacks. Our framework therefore allows us to capture existing PoW-based deployments as well as PoW blockchain variants that are instantiated with different parameters, and to objectively compare the tradeoffs between their performance and security provisions.}, booktitle={Proceedings of the 2016 ACM SIGSAC Conference on Computer and Communications Security}, publisher={Association for Computing Machinery}, author={Gervais, Arthur and Karame, Ghassan O. and Wüst, Karl and Glykantzis, Vasileios and Ritzdorf, Hubert and Capkun, Srdjan}, year={2016}, month={Oct}, pages={3–16}, collection={CCS ’16} }
 @article{Goldreich_Oren_1994, title={Definitions and properties of zero-knowledge proof systems}, volume={7}, ISSN={0933-2790, 1432-1378}, DOI={10.1007/BF00195207}, number={1}, journal={Journal of Cryptology}, author={Goldreich, Oded and Oren, Yair}, year={1994}, month={Dec}, pages={1–32} }
 @article{Haber_Stornetta_1991, title={How to time-stamp a digital document}, volume={3}, ISSN={1432-1378}, DOI={10.1007/BF00196791}, abstractNote={The prospect of a world in which all text, audio, picture, and video documents are in digital form on easily modifiable media raises the issue of how to certify when a document was created or last changed. The problem is to time-stamp the data, not the medium. We propose computationally practical procedures for digital time-stamping of such documents so that it is infeasible for a user either to back-date or to forward-date his document, even with the collusion of a time-stamping service. Our procedures maintain complete privacy of the documents themselves, and require no record-keeping by the time-stamping service.}, number={2}, journal={Journal of Cryptology}, author={Haber, Stuart and Stornetta, W. Scott}, year={1991}, month={Jan}, pages={99–111} }
 @inproceedings{He_Cui_Jiang_2019, title={An Improved Gossip Algorithm Based on Semi-Distributed Blockchain Network}, DOI={10.1109/CyberC.2019.00014}, abstractNote={With the continuous development of blockchain technology, more and more blockchain projects use semi-distributed P2P network structures. Although original gossip algorithm can be devoted to data synchronization in semi-distributed blockchain network, it can not be well applied to actual network environment. Since the probability of selecting a target node during data synchronization is fixed, it is inevitable that a message can be sent to a duplicate node. It will not only cause a lot of redundant messages, but also bring inefficient data synchronization. To address this problem, this paper proposes an improved HNA-Gossip algorithm which can reduce the probability of selecting duplicate nodes to send messages by recording historical node information dynamically. The simulation results show that, compared with the original gossip algorithm, various aspects of HNA-Gossip algorithm perform better.}, booktitle={2019 International Conference on Cyber-Enabled Distributed Computing and Knowledge Discovery (CyberC)}, author={He, Xiaowei and Cui, Yiju and Jiang, Yunchao}, year={2019}, month={Oct}, pages={24–27} }
 @article{Herlihy_2019, title={Blockchains from a distributed computing perspective}, volume={62}, ISSN={0001-0782}, DOI={10.1145/3209623}, abstractNote={The roots of blockchain technologies are deeply interwoven in distributed computing.}, number={2}, journal={Communications of the ACM}, author={Herlihy, Maurice}, year={2019}, month={Jan}, pages={78–85} }
 @inproceedings{Kedem_Silberschatz_1979, title={Controlling concurrency using locking protocols}, ISSN={0272-5428}, DOI={10.1109/SFCS.1979.12}, abstractNote={This paper is concerned with the problem of developing locking protocols for ensuring the consistency of database systems that are accessed concurrently by a number of independent transactions. It is assumed that the database is modelled by a directed acyclic graph whose vertices correspond to the database entities, and whose arcs correspond to certain locking restrictions. Several locking protocols are presented. The weak protocol is shown to ensure consistency and deadlock-freedom only for databases that are organized as trees. For the databases that are organized as directed acyclic graphs, the strong protocol is presented. Discussion of SHARED and EXCLUSIVE locks is also included.}, booktitle={20th Annual Symposium on Foundations of Computer Science (sfcs 1979)}, author={Kedem, Zvi and Silberschatz, Abraham}, year={1979}, month={Oct}, pages={274–285} }
 @article{Lamport_Shostak_Pease_1982, title={The Byzantine Generals Problem}, volume={4}, ISSN={0164-0925}, DOI={10.1145/357172.357176}, number={3}, journal={ACM Transactions on Programming Languages and Systems}, author={Lamport, Leslie and Shostak, Robert and Pease, Marshall}, year={1982}, month={Jul}, pages={382–401} }
 @article{Meneghetti_Parise_Sala_Taufer_2019, title={A survey on efficient parallelization of blockchain-based smart contracts}, url={http://arxiv.org/abs/1904.00731}, abstractNote={The main problem faced by smart contract platforms is the amount of time and computational power required to reach consensus. In a classical blockchain model, each operation is in fact performed by each node, both to update the status and to validate the results of the calculations performed by others. In this short survey we sketch some state-of-the-art approaches to obtain an efficient and scalable computation of smart contracts. Particular emphasis is given to sharding, a promising method that allows parallelization and therefore a more efficient management of the computational resources of the network.}, note={arXiv: 1904.00731}, journal={arXiv:1904.00731 [cs]}, author={Meneghetti, Alessio and Parise, Tommaso and Sala, Massimiliano and Taufer, Daniele}, year={2019}, month={Feb} }
 @article{Merkle_1978, title={Secure communications over insecure channels}, volume={21}, ISSN={0001-0782}, DOI={10.1145/359460.359473}, abstractNote={According to traditional conceptions of cryptographic security, it is necessary to transmit a key, by secret means, before encrypted massages can be sent securely. This paper shows that it is possible to select a key over open communications channels in such a fashion that communications security can be maintained. A method is described which forces any enemy to expend an amount of work which increases as the square of the work required of the two communicants to select the key. The method provides a logically new kind of protection against the passive eavesdropper. It suggests that further research on this topic will be highly rewarding, both in a theoretical and a practical sense.}, number={4}, journal={Communications of the ACM}, author={Merkle, Ralph C.}, year={1978}, month={Apr}, pages={294–299} }
 @inproceedings{Merkle_1988, place={Berlin, Heidelberg}, series={Lecture Notes in Computer Science}, title={A Digital Signature Based on a Conventional Encryption Function}, ISBN={978-3-540-48184-3}, DOI={10.1007/3-540-48184-2_32}, abstractNote={A new digital signature based only on a conventional encryption function (such as DES) is described which is as secure as the underlying encryption function -- the security does not depend on the difficulty of factoring and the high computational costs of modular arithmetic are avoided. The signature system can sign an unlimited number of messages, and the signature size increases logarithmically as a function of the number of messages signed. Signature size in a ‘typical’ system might range from a few hundred bytes to a few kilobytes, and generation of a signature might require a few hundred to a few thousand computations of the underlying conventional encryption function.}, booktitle={Advances in Cryptology — CRYPTO ’87}, publisher={Springer}, author={Merkle, Ralph C.}, editor={Pomerance, Carl}, year={1988}, pages={369–378}, collection={Lecture Notes in Computer Science} }
 @misc{Monica_California 90401-3208, title={Paul Baran and the Origins of the Internet}, url={https://www.rand.org/about/history/baran.html}, abstractNote={RAND researcher Paul Baran developed a solution that has evolved into one of the major technological innovations of our time.}, author={Monica, 1776 Main Street Santa and California 90401-3208} }
 @article{Morris_Wong_1985, title={Performance analysis of locking and optimistic concurrency control algorithms}, volume={5}, ISSN={0166-5316}, DOI={10.1016/0166-5316(85)90043-4}, abstractNote={New analytic models are presented which predict the maximum throughput of locking and optimistic concurrency control algorithms for a centralized database system. By making several simplifying assumptions, these models can be easily solved. The analytic results are tested against simulation and are shown to have an accuracy considerably better than some previously reported methods. The models are used to carry out a comparison between locking and optimistic control under stated assumptions. It is found that locking schemes consistently have higher maximum throughput than optimistic schemes.}, number={2}, journal={Performance Evaluation}, author={Morris, R. J. T and Wong, W. S}, year={1985}, month={May}, pages={105–118} }
 @article{Nakamoto, title={Bitcoin: A Peer-to-Peer Electronic Cash System}, abstractNote={A purely peer-to-peer version of electronic cash would allow online payments to be sent directly from one party to another without going through a financial institution. Digital signatures provide part of the solution, but the main benefits are lost if a trusted third party is still required to prevent double-spending. We propose a solution to the double-spending problem using a peer-to-peer network. The network timestamps transactions by hashing them into an ongoing chain of hash-based proof-of-work, forming a record that cannot be changed without redoing the proof-of-work. The longest chain not only serves as proof of the sequence of events witnessed, but proof that it came from the largest pool of CPU power. As long as a majority of CPU power is controlled by nodes that are not cooperating to attack the network, they’ll generate the longest chain and outpace attackers. The network itself requires minimal structure. Messages are broadcast on a best effort basis, and nodes can leave and rejoin the network at will, accepting the longest proof-of-work chain as proof of what happened while they were gone.}, author={Nakamoto, Satoshi}, pages={9} }
 @article{Pirrong_2019, title={Will Blockchain Be a Big Deal? Reasons for Caution}, volume={31}, ISSN={1745-6622}, DOI={10.1111/jacf.12379}, abstractNote={The initial enthusiasm for implementing blockchain in financial markets has been dampened considerably by its collision with economic realities. Though the author warns against avoiding the Panglossian trap of viewing ours as the best of all possible worlds, he reminds us that trusted institutions have evolved and emerged in a competitive environment as a means of economizing on transaction costs. Such costs arise from the nature of transactions, including crucially the information environment in which they take place. Though new technologies such as blockchain have the potential to reduce some of these costs, they often do so without fundamentally changing the underlying economic conditions that give rise to them. And as a result, institutions such as banks and exchanges that technologists scoff at may well prove surprisingly competitive and durable in the face of technological challengers. The most successful implementation of blockchain—Bitcoin—solves a very basic transactional challenge peculiar to cryptocurrency: the double spend problem. But it does so in a very expensive way, and many other transactions pose far more complex challenges. The three cautionary tales provided by the author—the first involving securities and derivatives trading and clearing, the second commodity trading, and the third proposals to “equitize” assets—all demonstrate the need to confront “Chesterton’s Fence” when evaluating the potential of blockchain in any particular application. In other words, to understand the value of a new technology for a given set of functions, one must understand the economic forces that have shaped the processes and institutions that currently perform those functions. When such forces are considered, it often becomes apparent that new technologies like blockchain will not prove superior to existing practices—and may even create adverse unintended consequences that offset and perhaps even eliminate its beneficial effects.}, number={4}, journal={Journal of Applied Corporate Finance}, author={Pirrong, Craig}, year={2019}, pages={98–104} }
 @article{Shrey_Singh_Sathya_Yogesh_2019, title={DiPETrans: A Framework for Distributed Parallel Execution of Transactions of Blocks in Blockchain}, url={http://arxiv.org/abs/1906.11721}, abstractNote={In most of the modern day blockchain, transactions are executed serially by both miners and validators; also, PoW is determined serially. The serial execution limits the system throughput and increases transaction acceptance latency, even unable to exploit the modern multi-core resources efficiently.}, note={arXiv: 1906.11721}, journal={arXiv:1906.11721 [cs]}, author={Shrey, Baheti and Singh, Anjana Parwat and Sathya, Peri and Yogesh, Simmhan}, year={2019}, month={Jun} }
 @book{Sompolinsky_Lewenberg_Zohar_2016, title={SPECTRE: A Fast and Scalable Cryptocurrency Protocol}, url={http://eprint.iacr.org/2016/1159}, abstractNote={A growing body of research on Bitcoin and other permissionless cryptocurrencies that utilize Nakamoto’s blockchain has shown that they do not easily scale to process a high throughput of transactions, or to quickly approve individual transactions; blocks must be kept small, and their creation rates must be kept low in order to allow nodes to reach consensus securely. As of today, Bitcoin processes a mere 3-7 transactions per second, and transaction confirmation takes at least several minutes. We present SPECTRE, a new protocol for the consensus core of cryptocurrencies that remains secure even under high throughput and fast confirmation times. At any throughput, SPECTRE is resilient to attackers with up to 50% of the computational power (up until the limit defined by network congestion and bandwidth constraints). SPECTRE can operate at high block creation rates, which implies that its transactions confirm in mere seconds (limited mostly by the round-trip-time in the network). Key to SPECTRE’s achievements is the fact that it satisfies weaker properties than classic consensus requires. In the conventional paradigm, the order between any two transactions must be decided and agreed upon by all non-corrupt nodes. In contrast, SPECTRE only satisfies this with respect to transactions performed by honest users. We observe that in the context of money, two conflicting payments that are published concurrently could only have been created by a dishonest user, hence we can afford to delay the acceptance of such transactions without harming the usability of the system. Our framework formalizes this weaker set of requirements for a cryptocurrency’s distributed ledger. We then provide a formal proof that SPECTRE satisfies these requirements.}, number={1159}, author={Sompolinsky, Yonatan and Lewenberg, Yoad and Zohar, Aviv}, year={2016} }
 @article{Stoll_Klaaßen_Gallersdörfer_2019, title={The Carbon Footprint of Bitcoin}, volume={3}, ISSN={2542-4785, 2542-4351}, DOI={10.1016/j.joule.2019.05.012}, number={7}, journal={Joule}, publisher={Elsevier}, author={Stoll, Christian and Klaaßen, Lena and Gallersdörfer, Ulrich}, year={2019}, month={Jul}, pages={1647–1661} }
 @misc{Vigna_2016, title={The Great Digital-Currency Debate: ‘New’ Ethereum Vs. Ethereum ‘Classic’}, ISSN={0099-9660}, url={https://blogs.wsj.com/moneybeat/2016/08/01/the-great-digital-currency-debate-new-ethereum-vs-ethereum-classic/}, abstractNote={The new digital currency Ethereum is only about three years old, but after a controversial software upgrade, it already has split in two.}, journal={WSJ}, author={Vigna, Paul}, year={2016}, month={Aug} }
 @article{Wang_Hoang_Hu_Xiong_Niyato_Wang_Wen_Kim_2019, title={A Survey on Consensus Mechanisms and Mining Strategy Management in Blockchain Networks}, volume={7}, ISSN={2169-3536}, DOI={10.1109/ACCESS.2019.2896108}, abstractNote={The past decade has witnessed the rapid evolution in blockchain technologies, which has attracted tremendous interests from both the research communities and industries. The blockchain network was originated from the Internet financial sector as a decentralized, immutable ledger system for transactional data ordering. Nowadays, it is envisioned as a powerful backbone/framework for decentralized data processing and data-driven self-organization in flat, open-access networks. In particular, the plausible characteristics of decentralization, immutability, and self-organization are primarily owing to the unique decentralized consensus mechanisms introduced by blockchain networks. This survey is motivated by the lack of a comprehensive literature review on the development of decentralized consensus mechanisms in blockchain networks. In this paper, we provide a systematic vision of the organization of blockchain networks. By emphasizing the unique characteristics of decentralized consensus in blockchain networks, our in-depth review of the state-of-the-art consensus protocols is focused on both the perspective of distributed consensus system design and the perspective of incentive mechanism design. From a game-theoretic point of view, we also provide a thorough review of the strategy adopted for self-organization by the individual nodes in the blockchain backbone networks. Consequently, we provide a comprehensive survey of the emerging applications of blockchain networks in a broad area of telecommunication. We highlight our special interest in how the consensus mechanisms impact these applications. Finally, we discuss several open issues in the protocol design for blockchain consensus and the related potential research directions.}, journal={IEEE Access}, author={Wang, Wenbo and Hoang, Dinh Thai and Hu, Peizhao and Xiong, Zehui and Niyato, Dusit and Wang, Ping and Wen, Yonggang and Kim, Dong In}, year={2019}, pages={22328–22370} }
 @article{Zhang_Lee_2019, title={Analysis of the main consensus protocols of blockchain}, ISSN={2405-9595}, url={http://www.sciencedirect.com/science/article/pii/S240595951930164X}, DOI={10.1016/j.icte.2019.08.001}, abstractNote={Blockchain is the core technology of many cryptocurrencies. Blockchain as a distributed ledger technology has received extensive research attention. In addition to cryptography and P2P (peer-to-peer) technology, consensus protocols are also a fundamental part of the blockchain technology. A good consensus protocol can guarantee the fault tolerance and security of the blockchain systems. The consensus protocols currently used in most blockchain systems can be broadly divided into two categories: the probabilistic-finality consensus protocols and the absolute-finality consensus protocols. This paper introduces some of the main consensus protocols of these two categories, and analyzes their strengths and weaknesses as well as the applicable blockchain types.}, journal={ICT Express}, author={Zhang, Shijie and Lee, Jong-Hyouk}, year={2019}, month={Aug} }
 @misc{ethereum/wiki, url={https://github.com/ethereum/wiki}, abstractNote={The Ethereum Wiki. Contribute to ethereum/wiki development by creating an account on GitHub.}, journal={GitHub} }
 @misc{The Effects of Concurrency Control on Database Management System Performance, url={https://apps.dtic.mil/sti/citations/ADA103141} }
