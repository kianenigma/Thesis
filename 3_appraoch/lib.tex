\chapter{A Novel Approaches Toward Concurrency in Blockchains} \label{chap:approach}

\begin{chapquote}{Official Go Programming Language Documentation.}
	Don't share memory to communicate, communicate to share memory.
\end{chapquote}

In this chapter we build up all the details and arguments needed to introduce our approach toward
concurrency within blockchains. We start with an interlude, enumerating different ways to improve
blockchains' throughput from an end-to-end perspective, and highlight concurrency as our method of
choice.

\section{Prelude: Speeding up a Blockchain - An Out-of-The-Box Overview}
\label{chap_approach:sec:ways_to_speedup}

Blockchains can be seen, in a very broad way, as \textit{decentralized state machines that
transition by means of transactions}. The throughput of a blockchain network, measured in
transactions per second, is a function of numerous components, and can be analysed from different
points of view. While in this work we focus mainly on one aspect, it is helpful to enumerate
different viewpoints and see how each of them affects the overall throughput \footnote{This
categorization is by no means exhaustive. We are naming only a handful.}.

\subsection{Consensus and Block Authoring}
We discussed how the consensus protocol provides the means of ensuring that all nodes have a
persistent view of the state (see Section \ref{chap_bg:subsec:consensus_authorship}), and it can
heavily contribute to the throughput of the system. Take, for example, two common consensus
protocols: Proof of \textbf{Work} and Proof of \textbf{Stake}. They use the computation power
(\textit{work}) and an amount of bonded tokens (\textit{stake}), respectively, as their guarantees
that an entity has \textit{authority} to perform some operation, such as authoring a block. It is
important to note that each of these consensus protocols has \textit{inherently} different
throughput characteristics \cite{meneghettiSurveyEfficientParallelization2019}. Proof of work, as
the names suggests, requires the author to prove their legitimacy by providing proof that they have
solved a particular hashing puzzle. This is slow by nature, and wastes a lot of computation power on
each node that wants to produce blocks, which in turn has a negative impact on the frequency of
blocks, which directly impacts the transaction throughput. Improving the throughput of Proof of Work
requires the network to agree on an easier puzzle, that can in turn make the system less secure
\cite{gervaisSecurityPerformanceProof2016} (further details of which are outside the scope of this
work).

To the contrary, Proof of Stake does not need puzzle solving, which is beneficial in terms of
computation resources. Moreover, since the chance of any node being the author is determined by
their stake\footnote{Using some hypothetical election algorithm which is irrelevant to this work.},
more frequent blocks do not impact the security of the chain as much as Proof of Work does.
Recently, we are seeing blockchains turning to verifiable random functions
\cite{dodisVerifiableRandomFunction2005} for block authoring, and deploying a traditional byzantine
fault tolerance voting scheme on top of it to ensure finality
\cite{buterinCasperFriendlyFinality2019, stewartPosterGRANDPAFinality2019}. This further
\textit{decouples block production and finality}, allowing production to proceed faster and with
even less drag from the rest of the consensus system.

All in all, one general approach towards increasing the throughput of a blockchain is to
\textit{re-think the consensus and block authoring mechanisms} that dictate when blocks are added to
the chain - specifically, at which frequency. It is crucially important to note that any approach in
this domain falls somewhere in the spectrum of centralized-decentralized, where most often
approaches that are more centralized are more capable of delivering better throughput, yet they may
not have some of the security and immutability guarantees of a blockchain. A prime example of this
was mentioned in table \ref{table:blockchain_types} of chapter
\ref{chap_bg:subsec:blockchain_types}, where private blockchains are named as being always the
fastest.

\subsubsection{Sharding}
An interesting consensus-related optimisation that is gaining a lot of relevance in recent years is
a technique called \textit{sharding}, borrowed from the databases field. Shards are slices of data
(in the database jargon) that are maintained on different nodes. In a blockchain system, shards
refer to sub-chains that are maintained by sub-networks of the whole system. In essence, instead of
keeping \textit{all} the nodes in the entire system synchronized at \textit{all} times, sharded
blockchains consist of \textit{multiple} smaller networks, each maintaining their own cannon chain.
Albeit, most of the time these sub-chains all have the same prefix and only differ in the most
recent blocks. At fixed intervals, sub-networks come to agreement and synchronize their shards with
one another. In some sense, sharding allows smaller sub-networks to progress
faster\cite{forestierBlockcliqueScalingBlockchains2019, al-bassamChainspaceShardedSmart2017,
shreyDiPETransFrameworkDistributed2019}.

\subsection{Chain Topology}

Another approach to improve the throughput is changing the nature of the chain itself. A classic
blockchain is theoretically limited due to its shape: a chain has only \textit{one} head, thus only
one new block can be added at each point in time. This property brings extra security, and make the
chain state easier to reason about (i.e. there is only one canonical chain). A radical approach is
to question this property and allow different blocks (or individual transactions) to be created at
the same time. Consequently, this approach turns a blockchain from a literal \textit{chain of
blocks} into a \textit{graph of transactions} \cite{pervezComparativeAnalysisDAGBased2018}. Most
often, such technologies are referred to Directed Acyclic Graphs (DAG) solutions. A prominent
example of this is the IOTA project\cite{mIOTANextGenerationBlock2018}.

Allowing the chain to grow from different heads (i.e. seeing it as a graph) allows true parallelism
\textit{at the block layer}, effectively increasing the throughput. Nonetheless, the security of
such approaches is still an active area of research, and achieving decentralization with such loose
authoring constraints has proven to be challenging \cite{sompolinskySPECTREFastScalable2016}.

Altering chain topology brings even more radical changes to the original idea of blockchain.
While being very promising for some fields such as massively large user applications (i.e. "Internet
of Things", micro-payments), we do not consider DAGs in this work. We choose to adhere to the
definitions provided in chapter \ref{chap:background} as our baseline of what a blockchain is.

\subsection{Deploying Concurrency over Transaction Processing}
\label{chap_approach:subsec:out_of_box_concurrency}

Finally, we can focus on the transaction-processing view of the blockchain, and try and deploy
concurrency on top of it, leaving the other aspects, such as consensus, unchanged and, more
importantly, \textit{generic}. This is very important, as it allows our approach (to be explained
further in this chapter) to be deployed on many chains, because it is independent of any
chain-specific detail. Any chain will eventually come to a point where it must execute some
transactions, be it in the form of a chain, or a DAG, with any consensus. Thus, concurrency is a
viable solution as long as the notions of transactions and blocks exist.

Our work specifically focuses on this aspect of blockchain systems, and proposes a novel approach
to achieve concurrency within each block's execution, both in the authoring phase and in the
validation phase, thereby increasing the throughput.

\subsection{Summary} \label{chap_bg:subsec:summary_speedup}

Each of these methods brings about an improvement in the throughput (in terms of transactions
processed per second) of the system in their own unique way. The question of \textit{which one will
be the dominant one} is too specific to a particular chain's assumptions, and outside the scope of
what we are trying to tackle in this work. As an example, for some consortium chains where consensus
is less of an issue, altering the consensus is likely to result in a significant throughput gain,
without using any of the other methods such as sharding and concurrency.

Instead, we acknowledge that each of these approaches have their own merit and could be useful in a
certain scenario. Therefore, we put our focus on concurrency for the rest of this work. We
concurrency as a universal improvement that can be applied to any chain. Moreover, It is worth
noting that having optimal hardware utilization (to reduce costs) is an important factor in the
blockchain industry, as many chains are ran by people who are making profit out of running
validators and miners.

Finally, by seeing these broad options, we can clarify our usage of the word "throughput". One might
notice that the first two options mentioned in this section (consensus, chain topology) can increase
the throughput at the \textit{block} level: more blocks can be added, thus more transaction
throughput. This is in contrast to what concurrency can do. The concurrency explained in
\ref{chap_approach:subsec:out_of_box_concurrency} is the matter of what happens \textit{within}
\textit{\textbf{a}} block. Henceforth, by throughput we mean throughput of transactions that are
being executed within a (\textit{single}) block. Similarly, by concurrent, we mean concurrent within
the transactions of a (\textit{single}) block, not concurrency between the blocks themselves.


\section{Concurrency within Block Production and Validation} \label{chap_approach:sec:concurrency}

In this section, we explain in detail how a concurrent blockchain should function. Most notably, we
define how a concurrent author and a concurrent validator differ from their sequential counterparts
by defining their requirements. Note that these requirements are mandatory and \textit{any} approach
toward concurrency in blockchains must respect them, as long as it has a common definition of a
blockchain as we named in chapter \ref{chap:background}. As a reader might expect based on previous
explanations, all of them boil down to one radical property: \textbf{determinism}.

To recap, the block author is the elected entity that proposes a new block consisting of
transactions. The block author must have already executed these transactions in some
protocol-specific order (e.g., sequentially), and noted the correct \textbf{state root} of the block
in its header. This block is then propagated over the network. All other nodes validate this block
and, if they all come to the same state root, they append it to their local chain. An author that
successfully creates a block gets rewarded for their work by the system.

\subsection{Concurrent Author} \label{chap_appraoch:subsec:concurrent_author}

We begin in a chronologically sensible way, with block authoring. Before anything interesting can
happen in a blockchain, someone has to author and propose a new block. Else, no state transition
happens.

A concurrent author has access to a pool\footnote{Also referred to as queue in
\ref{chap_bg:subsec:tx_queue}, and sometimes called \textit{mempool} in the industry.} of
transactions that have been received over the network, most often via the gossip protocol. Form a
consensus point of view, it is absolutely irrelevant to ensure all nodes have the same transactions
in their local pool. In other words, from a consensus point of view, there is no consensus in the
transaction pool layer. All that matters is that any node, once chosen to be the author, has a pool
from which it can choose transactions. Then, the author has a \textit{limited} amount of time to
prepare the block and propagate it.

A number of ambiguities arise here. We dismiss them all in the following enumeration to be able to
only focus on the concurrency aspect.

\begin{itemize}
	\item Typically, the author needs some way to \textit{prefer} a subset of the transactions pool,
	as most often all of it cannot be fit into the block. For this work, we leave this detail
	generic and assume that each author has first filtered out its \textit{pool} into a new
	\textit{queue} of transactions (noting that the former is \textit{unordered} and the latter is
	\textit{ordered}) that she prefers to include in the block. In reality, a common strategy here
	is to prioritize the transactions that will pay off the most fee, as this will benefit the block
	author.
	\item The fact that the authoring time is limited is the whole reason why the author is
	incentivized to use concurrency: the more transaction that can be fitted into the block in a
	limited amount of time, the more the sum of transaction fees, thus more reward for the author.
	\item A block must have some chain-specific \textit{resource} (e.g. computation, state
	read/write, byte length of transactions) limit. For simplicity, we assume that each block can
	fit a fixed \textit{maximum} number of transactions, but this limit is so high that the
	bottleneck is not the transaction limit itself, but rather the amount of \textit{time} that that
	author has to prepare the block, bolstering the importance of high throughput. In reality, some
	blockchains have adopted a similar approach (i.e., a cap on \textit{number} of transactions), or
	have limited the size of the block. Complex chains that support arbitrary code execution even go
	further and limit the \textit{computation} cost of the transactions - see, for example,
	Ethereum's gas metering\cite{perezBrokenMetreAttacking2020}.
\end{itemize}

Having all these parameters fixed, we can focus on the block building part, namely executing
each transaction and placing it in the block.

A \textbf{sequential author} would simply execute all the transactions one by one (in some order of
preference) up until the time limit, and calculate the new state root. These transactions are then
structured as a block. Concatenated with a header that notes the state root, the block is ready to
be propagated. The created block is an \textit{ordered} container for transactions, therefore it can
be trivially re-executed deterministically by validators, as long as everyone does it sequentially.

A \textbf{concurrent author's} goal is to execute these transactions in a concurrent way, hoping to
fit \textit{more} of them in the same limited \textit{time}, while still allowing the validators to
come to the same state root. This is challenging, because, most often, concurrency is
non-deterministic. Therefore, the author is expected to piggy-back some auxiliary information to its
block, to allow validators to execute it deterministically. Maintaining determinism is the first and
foremost criterion of the concurrent author.

Moreover, the secondary criterion is a net positive gain in throughput. The concurrent author
prefers to be able to execute more transactions within the fixed time frame that she has for
authoring, for she will then be rewarded with more transaction fees.

\subsection{Concurrent Validator} \label{chap_appraoch:subsec:concurrent_validator}

A validator's role is simpler in both the sequential and concurrent fashion. Recall that the a block
is an ordered container for transactions. Then, the sequential validator has a trivial role:
re-execute the transactions sequentially and compare state roots. The concurrent validator however,
is likely to need to do more.

More specifically, the concurrent validator knows that a concurrent author must have executed all or
some of the transactions within the block concurrently. Therefore, conflicting transactions must
have \textbf{preceded} one another in \textit{some} way. The goal of the concurrent validator is to
reproduce the \textit{same precedence} in an efficient manner, and thus arrive to the same state
root.

For example, assume the author uses a simple Mutex to perform concurrency. In this case, some
transactions inevitably have to wait for other transactions that accessed the same Mutex earlier.
This leads to an \textit{implicit} precedence between conflicting transactions. The author needs to
somehow transfer the precedence information to the validator, and the validator must respect this
order to arrive at the same state root.

\section{Existing Approaches}\label{chap_approach:existing}

In this section we look at some of the already existing approaches toward concurrency in
blockchains. These approaches are essentially a practical resemblance of what was explained in the
previous section. While doing so, we denote their deficiencies and build upon them to introduce our
approach.

\subsection*{Concurrency Control}

Every tool that we named for concurrency in chapter \ref{chap:background} can essentially be used in
blockchains as well, yet each takes a specific toll on the system in order to be feasible. All of
these approaches fall within the category of \textbf{concurrency control}. We begin with the
simplest one: locking.

A locking approach would divide the transactions into multiple threads\footnote{A 1:1 relation
between threads and transactions is also possible, given that the programming language supports
green threads.}. Each transaction within the thread, when attempting to access any key in the state
(Recall from \ref{chap_bg:subsec:kvdb} that the state is a key-value database), has to acquire a
lock for it. Once acquired, the transaction can access the key. This process is not deterministic.
Therefore, the runtime needs to keep track of which locks were requested by which thread, and the
\textit{order} in which they were granted. This information builds, in essence, a dependency graph.
This dependency graph needs to be sent to the validators as well. The validators parse the
dependency graph and, based on that information, spawn the required number of threads, and
distribute the transactions among them. \cite{dickersonSmartLocksAddingConcurrency2017} is among the
earliest works on concurrency within blockchain, and adopts such an approach.

The details of generating the dependency graph with minimum size, encoding it in the block in an
efficient way, and parsing it in the validator are being highly simplified here. These steps are
critical, as they are the main overheads of this approach. The size of this graph needs to be small,
as it needs to be added to the block and increases the network overhead. Moreover, the overhead of
this extra processing must be worthwhile for the author, as otherwise it would be in contrast to the
whole objective of deploying concurrency. There are some works that only focus on the "dependency
graph generating and processing" aspect of the process. They assume some means exist through which
the read and write set of each transaction can be computed (i.e., by monitoring the lock requests
that each thread sends at runtime). On top of this, they provide efficient ways to build the
dependency graph, and use it at the validator's end \cite{EnablingConcurrencySmart2018}.

The next step of this progression is to utilize transactional memory. This line of research follows
the same pattern. More recent works use software transactional memory (STM) to reduce the waiting
time and conflict rates. Similar to the locking approach, the runtime needs to keep track of the
dependencies and build a graph that encodes this information. Different flavours of STMs are used
and compared in this line of research, such as Read-Write STMs, Single-Version Object-based STMs,
and Multi-Version Object-based STMs
\cite{anjanaEfficientConcurrentExecution2019,anjanaEfficientFrameworkOptimistic2019}. Nonetheless,
the underlying procedure stays the same: some means of concurrency control to handle conflicts,
track dependency and use it to encode precedence, then re-create the same precedence in the
validator.

\subsection*{Concurrency Avoidance}

\nomenclature{Concurrency Avoidance}{A scheme in contrast to concurrency control in which no synchnization is used btween the threads.}

Next, we name an out-of-the box work that takes a rather different path. Many studies in the
blockchain literature use datasets from the database industry as their reference. Such datasets
might have unrealistically high rates of contention. \cite{saraphEmpiricalStudySpeculative2019} is
an empirical study that tries to determine the conflict rates within Ethereum transactions, a
\textit{live}, and arguably well-adopted network. While doing so, the study demonstrates a
different, wait-free approach. In the concurrent simulator of this work, all transaction are
executed in parallel, with the assumption that they do not conflict. If a conflict happens and a
transaction aborts, it is discarded, and re-executed again at later phase, sequentially. This
essentially clusters transactions into two groups: concurrent and sequential. All of the concurrent
transactions are guaranteed not to conflict. The sequential transaction do not matter as they are
executed sequentially. Aside from their findings about the conflict rates in different periods of
time in Ethereum, they also report speedups in \textit{some} cases, not being too shy of the speedup
amounts reported by \cite{dickersonSmartLocksAddingConcurrency2017}, which uses locking.

This is an inspiring finding, implying that perhaps complicated concurrency control might not be
needed after all for many of the transactions in \textit{some} period of time, based on the
contention of the transactions. In some sense, this work adopts a technique that we coin as
\textbf{concurrency avoidance}, instead of \textbf{concurrency control}. As a consequence, the
system need not deal with conflicts in any way, because they are rejected and dealt with separately
in the sequential phase.

\subsection*{Static Analysis}

Finally, we note that there has also been \textit{some} work on pure static analysis in the field of
blockchains, yet all of those that we have found require fundamental changes to the programmable
language of the target chain. For example, \cite{bartolettiTrueConcurrentModel2019} provides an
extension to the Ethereum's smart contract language, Solidity, that allows it to be executed in a
truly concurrent manner (by essentially limiting the features of the language). Similarly, RChain is
an industrial example of a chain that has a programming model that is fundamentally
concurrent\cite{darrylRCast21Currency2019}, namely
pi-calculus\cite{turnerPolymorphicPiCalculusTheory1996}. Such approaches are also inspiring, yet we
prefer devising an approach which does \textit{not} need to alter such fundamental assumptions about
the programming model, in favour of easy adoption and outreach.

The aforementioned 3 broad approaches (concurrency control, concurrency avoidance, static analysis)
are essentially the taxonomy that we have found to answer the first research question, namely the
different ways to utilize concurrency within a blockchain. We close this section with a remark about
the scope of the referenced works in this chapter

\begin{remark}
	Most of the surveyed related work name their work as approaches toward concurrency for
	\textbf{smart contracts}. At this point, it would be helpful to clarify that. The details of
	smart contract chains are well beyond the scope of this work. But, it is worth noting that a
	smart contract chain is a \textit{fixed} chain that has a fixed state transition logic, and a
	part of that logic is to store codes (smart contracts) and execute them upon being dispatched.
	Moreover, since Ethereum is the prominent smart contract chain, all of these works present
	themselves with simulators that can hypothetically be implemented in the Ethereum node. To the
	contrary, we do not limit ourselves to smart contracts or any specific chain in this work;
	instead, we build upon the idea that the future of blockchains will not be a \textit{single}
	chain (chain maximalism), but rather an abundance of domain specific chains interoperating with
	one another. To achieve this, one needs to think in the context of a framework for building
	blockchains, not a particular blockchain per se.
\end{remark}

\section{Our Approach} \label{chap_desgin:sec:our_approach}

In this section, we describe our approach towards concurrency in a blockchain runtime, both in the
authoring phase and in validation. First, we begin by drawing a conclusion from the surveyed studies
in \ref{chap_approach:existing}.

\subsection{Core Idea: Finding a New Balance}

We begin by pointing out that all of the mentioned works, regardless of their outcome, produce a
sizable amount of overhead. We think these overheads are preventable and argue that perhaps a different
approach can prevent them all-together.

Both locking and transactional memory result in a sizeable overhead while authoring. This is mostly
hidden in some sort of runtime overhead, for example, the need to keep track of locking order, and
consequently to parse it into a dependency graph. Moreover, this dependency graph inevitably
increases the block size, because the dependency graph needs to be propagated to all other nodes.
Finally, the validator also needs to tolerate the overhead of parsing the dependency graph and
making informed, potentially complicated decisions based upon it. These are all overheads compared
to the basic sequential model. In essence, we express skepticism toward these complex runtime
machinery to deal with conflicts, and record precedence.

On the other hand, the pure \textit{concurrency avoidance} model is likely to fail under any
workload with some non-negligible degree of contention, because it basically falls back to the
sequential model where most transactions are aborted and moved to the sequential model. Results from
\cite{saraphEmpiricalStudySpeculative2019} show the same trend.

To the contrary, we aim to minimize these overheads by finding a new balance between the
"concurrency control" and "concurrency avoidance" models. Moreover, we find a new balance
between "runtime" and "static" as well. While \textit{some} runtime apparatus is needed to orchestrate the
execution and prevent chaos, tracking all dependencies is likely to be too much. Similarly, while a
purely static approach toward concurrency is a radical change to the programmable language of a chain, we
claim that \textit{some} static information could nudge the runtime to the right path.

\subsection{Key Ideas: Almost Wait-Free, Delegation and Pseudo-Static}

Our approach is based on three key pivotal ideas, explained in the next sections.

\subsubsection{Almost Wait-Free: "Taintable" State}\label{chapt_approach:subsubsec:taintable_state}

\nomenclature{Taintable State}{Special HashMap-like datastructure that assigns a taint to each key.}
We have already seen that locking is a common primitive to achieve shared-state concurrency. In our
approach, we relax this primitive such that any access to a shared state by a thread \textbf{does
not incur long waiting times}, but instead, it might \textit{immediately} fail. To do so, we link
each key in the state database with a \textbf{taint} value. If a key has never been accessed before,
it is untainted. Once it is accessed by any thread (regardless of the type of operation being read
or write), it is tainted by the identifier of that accessor thread. Henceforth, any access to this
key by any other thread fails, returning the identifier of the original tainter (aka,
\textit{owner}) of the key. As we shall see in the implementation (see \ref{chap:impl}), this
approach is \textit{almost} wait-free, meaning that threads almost always proceed immediately with
any state operation. Indeed, a thread can always freely access keys that it has already tainted
before.

\subsubsection{Not Control, Nor Avoidance: Concurrency Delegation}

If a thread, in the process of executing a transaction, tries to access a tainted state key, it
forwards this transaction to the owner of that key. By doing so, a thread basically
\textit{delegates} the task of executing a transaction concurrently to another thread because it
cannot meet the state-access requirements of the transactions itself; based on the available error
information, the recipient (i.e., the owner of the failing state key) is more likely to be capable
of doing that. This is the middle ground between \textit{concurrency avoidance} and
\textit{concurrency control}, which we have coined \textit{concurrency delegation}.

Compared to concurrency control, threads in the concurrency delegation model do not try to resolve
contention in any sophisticated way. Instead, they simply delegate (aka, \textit{forward}) the
transactions to whoever they think \textit{might} be able to execute them successfully. There is no
waiting involved, and no record being kept about access precedence.

On the other hand, compared with concurrency avoidance, concurrency delegation does not
automatically assume that just because a transaction's state operation has failed, the transaction
cannot be executed in any concurrent way. Instead, the transaction might be executable by another
thread, which is known to be the owner of the problematic state key. Therefore, instead of
immediately being discarded (and potentially executed sequentially at the end), each transaction is
given a second chance to succeed in a concurrent fashion.

Finally, if a transaction is already forwarded and still cannot be executed, (only) then it is
forwarded to a sequential queue to be dealt with later. We name such transactions \textbf{orphans}.
This implies that each transaction is at most forwarded twice: once to another potential recipient,
and next, when needed, to the be declared as orphan.


Why is "a failing transaction that has already been forwarded once" considered an orphan? We
present an example to make this clearer: assume thread $T2$ receives a forwarded transaction from
$T1$. Based on the the protocol of delegation, we know that this transaction must have at least
one key which is owned by $T2$, because $T1$ failed to access it and thus forwarded it to $T2$.
Moreover, if the transaction still fails, it means that it has at least one other key that is
owned by some other thread $T3$. This implies that $T2$ is not able to execute this
transaction, and any further recipient of this transaction, including $T3$, will also fail to
execute it because the transaction has at least one key owned by $T2$.

Therefore, no transaction ever needs to be forwarded more than once. A transaction can be
forwarded at most once, and thereafter it is considered to be an orphan.

\subsubsection{Pseudo-Static}

We predict that concurrency delegation works well when threads need to seldom forward transactions
to one another. In the delegation model, the transactions need to be \textit{initially} distributed between
the threads in some educated and effective way to minimize forwarding. This is where we turn to
\textit{pseudo-static heuristics} in the form of a hint. We use the term "\textit{pseudo-static}"
because we do not mean information that is necessarily available at compile time, but is rather
known \textit{before} a particular transactions is executed. In other words, the information is
\textit{static with respect to the lifetime of a transaction} and can be inferred without the need
to \textit{execute} the transaction, but rather by just \textit{inspecting} it.

Our approach is based upon these key ideas: taintable state, concurrency delegation, and
pseudo-static hints. In the next section, we connect these ideas and depict how they work together
in the form of a unified algorithm.

\subsection{Baseline Algorithm} \label{subsec:baseline_alg}

We first look into the baseline algorithm, which essentially combines the concurrency delegation and
the taintable state, without any static heuristics. We consider a single master thread (henceforth
called "master"), and $4$\footnote{Needless to say, all of the arguments in this section are
applicable to virtually any number of threads.} worker threads (henceforth called "workers"),
each having the ability send messages over channels to one another. The master has access to a
potentially unbounded queue of (\textit{ordered}\footnote{Recall that each node has an unordered
\textit{pool} of transaction, from which they chose an ordered \textit{queue} based on some
arbitrary preference.}, ready to execute) transactions. Moreover, it has an (initially) empty queue
of orphan transactions, to which workers can forward transactions. Additionally, each worker has a
local queue, to which the master and other workers can send transactions. The master and all workers
share a reference to the same state database, $S$, which follows the logic of a taintable state as
explained in \ref{chapt_approach:subsubsec:taintable_state}.

\begin{remark}
	As we will see, it is crucial to remember that both the transaction queue and the block are
	\textbf{ordered} containers for transactions. In other words, they both act like an ordered
	\texttt{list}/\texttt{array}/\texttt{vector} of transactions. The order of transaction in the
	queue will end up being used to order the transaction in the final block as well.
\end{remark}

The master's (simplified) execution logic during \textbf{authoring} is as follows:

\begin{enumerate}
	\item \textbf{Distribution phase}: The master starts distributing transactions between workers
	by some arbitrary function $F$. In essence, $F$ is a $fn(transaction) \rightarrow identifier$,
	meaning that, for each transaction, it outputs one thread identifier. Once the distribution is
	done, each transaction in the queue is \textit{tagged} by the identifier of one worker thread.
	The distribution phase ends with the master sending each transaction to its corresponding
	worker's local queue.

	\item \textbf{Collection phase}: The master then waits for reports from all workers, indicating
	that they are done with executing all of the transactions that they have received earlier.
	During this phase, threads might forward transactions to one another, and might forward
	transactions back to the master, if deemed to be orphan, exactly as explained in the concurrency
	delegation model. Both of these events are reported to the master and the tag of each
	transaction might change in the initial queue. Once termination is detected by the master
	thread, a message is sent to all workers to shut them down.

	\item \textbf{Orphan phase}: Once all worker threads are done, the master executes any
	transactions that it has received in its orphan queue. At this point, the master thread is sure
	that there are no other active threads in the system, thus accessing $S$ without worrying about
	the taint is safe. The transactions are executed sequentially on top of $S$, and their tag is
	changed to a special identifier for orphan transactions.
\end{enumerate}

Then, the master is ready to finalize the block. By this point, each transaction is either tagged to
be orphan, or with the identifier of one of the $4$ worker threads. In essence, we have clustered
transactions into $4 + 1$ \textit{ordered} groups. The validator who receives this block respects
this clustering and executes transactions with the same tag in the same thread. In the first $4$
groups, the transactions within a group might conflict with one another (e.g. attempt to write to
the same key), but they are ordered and are known the be executed by the same thread, thus
deterministic. The transactions in the last group, namely the orphan group, are executed
sequentially and in isolation, thus deterministic.

The worker side of the processing (with slght simplifications) is as follows:

\begin{enumerate}
	\item \textbf{Depleting local queue}: Having received a number of transactions from the master
	(after the "\textit{Distribution phase}"), each worker then tries to deplete its local queue.
	For each transaction $Tx$, the logic is as follows:

	If $Tx$ is executed successfully, nothing is done or reported. This is because the master is
	already \textit{assuming} that $Tx$ is executed by the current worker, therefore nothing need to
	be reported. If $Tx$ fails due to a taint error, it is forward to the owner of the state key
	that caused the failure. Note that at this point we know that $Tx$ has not been forwarded
	before, because it is being retrieved from the initial local queue.

	At the end of this phase, the worker sends an overall report to the master, noting how many
	of the transactions it could execute successfully, and how many ended up being forwarded. This
	data is then used at the master to detect termination.

	\item \textbf{Termination phase}: Once done with their local queue, the workers listen for
	two types of messages, namely \textit{termination} or \textit{forwarded} transactions from other threads.
	Termination is the message from the master to shut down the worker. Forwarded transactions are
	those that another worker is \textit{delegating}/\textit{forwarding} to the current
	worker because of a taint error. The forwarded transaction is then executed locally and, if it is
	successful, the result is reported to the master. If the execution fails again due to a taint
	error, then the transaction is forwarded to the master as an orphan.

	Note that in this case, reporting is vital, because a worker is ending up executing a
	transaction and the master is \textit{not} aware of it, because the worker who finally executes the transaction is not the same as
	assigned in the "Distribution phase" of the master. This reporting is also needed for the termination
	detection of this phase.
\end{enumerate}

\subsubsection{Analysis and Comments on the Baseline Algorithm} \label{chap_approach:subsec:comment_baseline}

A number of noteworthy remarks exists on the baseline algorithm:

\textbf{Termination Detection}. We intentionally did not describe how the master detects the
termination of the collection phase, because, if we wanted to describe it, we needed further information from the
worker's logic. Recall that the master knows how many transactions it has initially
distributed between all the workers. Moreover, from the reports sent by the worker at the end of
"Depleting local queue" phase, it knows how many of them executed in their \textit{designated
thread}, and how many of them ended up being \textit{forwarded}. Also, recall that each forwarded
transaction, upon being executed successfully, is reported to the master. Similarly, each forwarded
transaction that fails is also reported to the master (by being forwarded to the orphan queue
residing in the master thread). Thus, the master can safely assert that \textit{Termination is achieved once
the sum of "all locally-executed transactions at workers", "forwarded and successfully executed
transactions" and "orphan transactions" is equal to the initial count of transactions in the queue}.
At this point, the termination message is created and broadcasted to all workers.

\begin{definition}
	\textbf{Termination of the master thread's collection phase. }

	Assume $N$ initial transaction. Each worker thread, upon finishing the "deplete local queue"
	phase, reports back $\{ r_{1}, r_{2}, r_{3}, r_{4} \}$ respectively, indicating the number of transactions
	that each executed locally. Given $D$ as the number of reports of transactions being
	forwarded, and $O$ as the size of the orphan queue, termination of the collection phase is achieved iff

	\begin{equation}
		\sum_{t = 1}^{4} r_{t} + D + O == N
	\end{equation}
\end{definition}


\textbf{Maintaining Order}: Aside from termination detection, it is also vital for determinism that
the master takes action upon the report of a transaction being forwarded. This is because
\textbf{once the tag of a transaction changes, it is likely that its order must also change within
the queue}. For example, if a transaction, initially assigned to $T_{1}$ is known to be forwarded
and executed by $T_{2}$, it is important to re-order it in the initial transaction queue such that
it is placed \textit{after} all the transactions initially assigned to $T_{2}$. This is because, in
reality, $T_{2}$ \textit{first} executed all of its designated transactions and \textit{then}
executes any forwarded transactions. Recall that the queue is an ordered container for transactions
and its order will eventually end up building the order of the transaction in the block.

\textbf{Orphan Transactions}. An orphan transaction is a transaction that has already been
forwarded and still fails to execute at its current host thread, due to a taint error. We now
represent this from a different perspective. In our concurrency delegation scheme, threads race to
access state keys and, upon successful access, they taint them. Any transaction has a number of state
keys that it needs to access in order to be processed. \textit{An orphan transaction is one that has
state keys being tainted by at least two \textbf{different} threads}. For example, assume a
transaction needs to access keys $K_{1}$ and $K_{2}$. Assume thread $T_{0}$ is executing this
transaction. If $K_{1}$ is already tainted by a $T_{1}$ and $K_{2}$ by a $T_{2}$, then this
transaction will inevitably end up being in the orphan queue. The transaction is first forwarded
from $T_{0}$ to $T_{1}$, where it can successfully access $K_{1}$, but still fails to access
$K_{2}$ and thus orphaned\footnote{An interesting optimization can be applied on top of this logic,
which is explained further in \ref{chap_impl:sec:opt_orph}.}.

\textbf{Minimal Overhead}. Our approach incurs minimal overhead to the block. In
fact, the only additional data needed is one identifier attached to each
transaction, indicating which thread must execute it (and the special case thereof, orphan
transaction), namely the \textit{tag}. This can be as small as a single byte per transaction,
which is negligible. Note that transactions within the block still maintain partial order: the
transactions of a particular tag are sorted within their tag. Only the relative order of
transactions from different tags is lost, which is not significant, because they are guaranteed by
the author to not conflict.

\textbf{Validation}. We can now consider validation as well. As expected, due to the minimal
overhead and the simplicity of the baseline algorithm, the validation logic is fairly simple. Each
block is received with all of its transactions having a tag. The ones tagged to be orphan are set
aside for later execution. Then, one worker thread is spawned per tag and transactions are assigned
to threads based on their tags, and in the same relative order. The workers can then execute
concurrently, \textit{without the need for any concurrency control}, because they effectively know
that all contentious transactions already have the same tags, and thus are ordered
\textit{sequentially} \textit{within that tag/thread}. In essence, the validation is fully
\textit{parallelizable} and does not need any synchronization. Once all threads are finished, the
orphan transactions are executed sequentially and validation comes to an end.

\subsubsection{Determinism in Concurrency Delegation}

Finally, we must address the most important requirement: \textbf{Determinism}. We proof determinism by
showing that the validation and authoring both have the exact same execution environment, and the
transactions are executed in the exact same order in both phases. In more detail, both the author and the
validator spawn the same number of threads, and all of the transactions executed by any
thread during authoring are re-executed by a single thread in the validation phase as well.

First, consider all transactions that are executed in their initially designated worker, based on
the aforementioned distributor function $F$. All of these transactions are assigned a tag, and have
some partial order within the tag. By design, they are also placed in their
designated worker thread's queue with the same order, and thus executed in the same order.
Consequently, they are, yet again, placed in the final block with the same order within the tag.
Therefore, the worker thread in the authoring phase and the worker thread
in the validation phase will execute \textit{exactly the same} transactions, in the exact same
order.

Next, consider the transactions that ended up being forwarded. These transactions are
executed \textit{after} the designated transactions of their final host thread. The master thread,
responsible for building the final block, has to note this change and ensure that this partial
order is maintained in the final block. The first step for the master is to change the tag of this
forwarded transaction to the tag of its newly designated thread. Then, to ensure that the order
\textit{within the tag} is maintained, the master simply places this transaction at the end
of the queue. This ensures that this transaction will be placed in the final block in such a way
that it is executed after all the transactions that have the same tag. Then, we can apply the same
logic as the previous paragraph and assert that the order of execution of all transactions with a
specific tag stays the same within one thread in both authoring and validation, thus deterministic.

Last but not least, the orphan transactions also need the same property to be executed deterministically:
maintaining order. The master needs to make sure that all the transactions that are tagged as
\texttt{Orphan} within the block have the same order as they were executed.

Given the deterministic execution of all types of transactions in our concurrent system, we conclude that our approach is fully deterministic, with minimal additional effort
effort. Indeed, the only subtlety that needs to be taken care of is the re-ordering of a transaction in
the queue (and consequently the block) when it is \textit{successfully} forwarded.


\subsection{Applying Pseudo-Static Heuristics} \label{subsec:applying_static_hints}

The previous section is a complete description of a concurrent system that is deterministic and can be deployed
as-is, without any further requirements. Nonetheless, one can argue that this system might not be
efficient in the throughput gain that it can deliver. This is because we said nothing about the
distribution function of the master, namely $F$. While we keep this distribution function generic in
this entire work, we make one important claim about it: using static information of
the transaction can be \textit{very} beneficial, and is key to high performance.

Static information is anything that can be known about a transaction, before it hits the runtime
and gets executed. In other words, we are interested in any information that can be inferred from
the transaction, \textit{without} needing to \textit{execute} it. A clear example of such information is the
origin of the transaction (i.e., the sender account). Because of public key cryptography usage, all
transactions carry a signature and the origin account as a part of their payload. Therefore, using
the origin as static information is permitted, because it is known even \textit{before} the
transaction is executed. The similar reasoning applies to the arguments of the transaction as well.
These are information that are encoded in the payload of the transaction, and the runtime of the
master thread (before starting authoring) can effectively use them to optimise $F$. To the contrary,
consider the return value of the transaction. This is a piece of information that is not considered
static with respect to the transaction, because it can only be known by executing the transaction.

\begin{remark}
	Given the above paragraph, we denote that out definition of static is different than the term
	which is usually referred to compile-time information. Therefore, we used the term pseudo-static
	in some places to delineate the difference.
\end{remark}

We use this these pseudo-static information to our benefit, by proposing a static annotation to be added by the programmer
to each transaction. Recall that the underlying state of the blockchain runtime is a key-value
database, so each state access is linked to a key. Moreover, if $F$ can know the list of keys
accessed by each transaction, it could create a perfect distribution where no transaction is ever
forwarded or orphaned, because no thread ever reaches a taint error while accessing the state. Of
course, things are not simple. In practice, it is impossible to know the execution path of a complex
transaction without executing it\footnote{An interested reader can refer to the "Halting problem"
for a formal representation of this issue \cite{burkholderHaltingProblem1987}.}, therefore knowing
the exact state keys that it must access is impossible.

The important point is to remember that the annotation can still be reasonably accurate without the
need for executing the transaction. This annotation can state, in a best effort manner, which state
keys are \textit{likely} to be accessed by this transaction, based on, for example, one of the
common execution paths of the transaction. Listing \ref{lst:static_pseudo} shows an abstract example of these static \textbf{annotations}.

\begin{lstlisting}[caption={Example of Static Hints}\label{lst:static_pseudo}]
#[access = (origin.state_key(), arg1.state_key())]
fn transaction(origin, arg1, arg2) {
	read(origin);
	if condition {
		// more probable branch!
		read(arg1)
	} else {
		// less probable branch!
		read(arg2)
	}
}
\end{lstlisting}

We observe a transaction, embodied as a function named \texttt{transaction}, which has 3 arguments:
the origin and two auxilliary ones. Furthermore, we observe a macro\footnote{This style refers to the
Rust programming language, but is applicable to any other compiled language as well.} that is
providing the state keys that \textit{might} be accessed by this transaction. In this case, the
second argument is not relevant and seemingly only some state key of the origin and the
\texttt{arg1} might be accessed.

This macro syntax is just one example - its specification is irrelevant at this point. What matters is that the transaction can provide the runtime with some easy-to-compute, pseudo-static
information about the \textit{state access} of the transaction (therefore we called the macro
\texttt{\#access} in listing \ref{lst:static_pseudo}). The runtime can then effectively use this
information in the transaction distribution phase to come up with a better distribution that leads
to less contention, and, consequently, to less forwarding and less orphans.

\begin{remark}
	One might question: why is the system limited by the halting problem and cannot pre-execute all transactions in the first
	place? This is partially answered in the transaction validation section (see
	\ref{chap_bg:subsec:validation}). The more detailed reason is that this static hinting must fit
	into the transaction validation pipeline, because the transactions for which we are keen to know
	the access keys are not yet included in any block, thus not accountable. In short, executing all
	transactions in order pool is an easy Denial Of Service attack vector, specially in chains that
	allow arbitrary code execution by the user.
\end{remark}

\subsection{System Architecture}

In this section, we bring \ref{subsec:baseline_alg} and \ref{subsec:applying_static_hints} into one
picture, and explain the entire system architecture. The all-in-one system architecture is presented
in Figure \ref{figure:figures/design.png}. In the following paragraph, we explain all its
components, as indicated by the labels in the figure, finally connecting all the dots of our
concurrent system as we move forward.

\todo[inline]{so I have to update refs to all stuff to Figure~}

\figuremacro{figures/design.png}{Overall System Architecture.}

%Figure \ref{figure:figures/design.png} includes enumerated labels on each system component, and we use these labels to explain each component.

\begin{itemize}
	\item \textbf{1} shows the taintable state. We can see that each cell is not merely a key-value
	pair, but rather a \texttt{(taint, key, value)} triplet. Some keys are not tainted
	(\texttt{None}), while some are already tainted by a thread because they have been accessed by
	it. An access to each of the keys by a thread succeeds if the key is untainted, or is tainted by
	the accessor thread. Else, an error is returned, notifying the identifier of the thread that owns
	the key.
	\item \textbf{2} emphasizes the need of all threads to access the state; ideally, this access is wait-free. Any access to a key in the
	state either \textit{immediately} succeeds, in which case the key is tainted. Else, it is
	\textit{immediately} rejected because of a taint error.
	\item \textbf{3} shows the communication channels between the master and workers. These channels
	allow all of the threads to communicate by means of sending messages to one another. Most
	often these messages are simply a transaction to be forwarded. It is worth noting that worker
	threads can also communicate in the same manner.
	\item \textbf{4} indicates the local queue of each thread. This is where the transactions that the
	master thread designates to each thread live, until they are executed. Needless to say, this
	queue is also ordered.
	\item \textbf{5} identifies the orphan queue, where the master thread maintains an ordered queue of
	transactions that are essentially rejected by the workers and need to be executed in a second
	sequential phase. Once all the workers are done accessing the state, the master thread
	exclusively starts using the state (essentially ignoring all the taint values) and executes
	all the orphan transactions sequentially.
	\item \textbf{6} importantly depicts the transaction distributor component of the master thread.
	This component is invoked prior to the process of authoring to tag all the transactions that are
	ready to be executed in the transaction queue. This effectively determines which thread gets to
	execute which transactions.
	\item \textbf{7} is the transaction queue, where an ordered subset of the transaction pool is
	verified and awaiting to be delegated to worker threads.
\end{itemize}

With this overall blueprint of the system's architecture, we conclude the design of SonicChain, and move forward to the next chapter, where
we cover some of the implementation details of our prototype.






