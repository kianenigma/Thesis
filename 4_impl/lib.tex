\chapter{Implementation} \label{chap:impl}

\begin{chapquote}{u/goofbe on reddit}
	Rust is like a futuristic laser gun with an almost AI-like foot detector that turns the safety
	on when it recognises your foot.
\end{chapquote}

In this chapter we bring the system architecture mentioned at the end of the chapter
\ref{chap:approach} one step closer to a real programming environment. This chapter is by no means
extensive, since there are many many implementation details that could be worth noting. Nonetheless,
in favour of brevity we minimize the details to only those that:

\begin{itemize}
	\item Are important with regards to the evaluation of the system.
	\item Impose a particular practical challenge that we find interesting.
\end{itemize}

For the implementation, we use the Rust programming language
\cite{klabnikRustProgrammingLanguage2019}. Being backed by Mozilla, Rust has a lot to offer in the
domain of system programming and low-level applications, such as a blockchain runtime. Rust is a
unique language, among the few rare ones in which one can claim that the learning curve is indeed
steep, as mastering it is \textit{not} just a matter of learning the new syntax. One of the reasons for this
learning curve is Rust's compile-time memory management system, which means all allocations and
de-allocations are inferred and checked \textit{at compile time}. This ensures that the program is
memory-safe, whilst having no garbage collector at runtime. Rust delivers, in some sense, the performance
of C, combined with the abstractions and safety features of Java or
C\#\cite{jungRustBeltSecuringFoundations2017}.

\begin{remark}
	We assume some basic Rust knowledge in the rest of this chapter. An interested reader without
	any Rust experience can also follow, yet they are likely to have to look up some types and
	concepts in the documentation.
\end{remark}

Lastly, it is worth mentioning that our choice is not merely out of interest. In the last few years,
Rust have been heavily invested in, by big blockchains companies and in the field of
research\cite{RustBlockchain}.

\section{The Rust Standard Library}

First, we explain some of the primitive types available in the Rust's standard library that we
use in our implementation. Note that these types are merely \textit{our choice} for this implementation. Although similar data
types from other libraries (or \textit{crates}, in the Rust jargon) are also acceptable, we prefer
to limit ourselves to the standard library and remain dependency-free in our proof-of-concept
implementation.

\subsection{State: HashMap and Locks} \label{chap:impl:subsec:state_and_concurrent_map}

For the taintable state, we need a data type that is \textit{similar} to a typical concurrent
\texttt{HashMap}\cite{barnatFastDynamicallySizedConcurrent2015}, yet has slight differences. Rust
does not provide a concurrent HashMap in the standard library, so the way to go is to
implement our own custom data structure. To implement this, we use a \texttt{HashMap} and a
\texttt{RwLock}, both of which are provided by the standard library. The HashMap behaves just as a
typical \texttt{HashMap} in any programming language. The \texttt{RwLock} is a locking primitive
with read-write distinction, where multiple \textit{read} requests can be done at the same time,
while a \textit{write} request will block access of all-but-one requests.

Like most data types in languages that support generics, Rust's default \texttt{HashMap} is
generic over both the key and value type that it uses. The final \texttt{HashMap} is using opaque
byte arrays as both the key and value type. For the keys, we do our own hashing and concatenation to
compute the location (i.e., the final key) of any given state variable. For example, to compute the
key of where the balance of an account is stored, we compute: \texttt{"balances:balance\_of".hash()
+ accountId.hash()} and use the final byte array as the key. As for the values, to be able to store
values of different types in the same map, we encode all values to a binary format, thus, a byte
array is used.

\begin{lstlisting}[caption={Key and Value Types (with slight simplification -- see \ref{lst:state_final}). }\label{lst:maps}]
/// The key type.
type Key = Vec<u8>;
/// The value type.
type Value = Vec<u8>;

/// Final State layout.
type State = HashMap<Key, Value>;
\end{lstlisting}

Listing \ref{lst:mpsc} shows how the final state type is created. Namely, we alias the state to be a hashmap with both key and values being opaque byte arrays.

\begin{remark}
	A careful reader might notice at this point that we explained how we build the key byte array
	(hashing and concatenation), yet we have not yet discussed how the value byte array is built.
	This will be explained in \ref{chap_impl:sec:bonus}.
\end{remark}

\subsection{Threading Model}

In short, Rust's threading model is \texttt{1:1}: each thread will be mapped to an
operating system thread (which sometimes directly maps to a hardware thread). Naturally, this means
that the overhead of spawning new threads is not negligible, but there are no additional runtime
overheads. This justifies using a small number of threads and assigning a large number of
transactions to each, rather than, for example, creating one thread per transaction.

The main purpose of this decision is for Rust to remain a \textit{runtime-free} programming
language\cite{RustJourneyAsync}, meaning that there is zero-to-minimal runtime in the final binary.
The opposite of the \texttt{1:1} threading model is the \texttt{N:M} model, also known as
\textit{green threads}. Such threads are lightweight, and do not map to a hardware/operating-system
thread in any way. Instead, they are a software abstraction and are handled by a
runtime\footnote{tokio is one of the best known such runtimes in the Rust
ecosystem\cite{TokioRust}.}. Therefore, a language that wants to support green threads needs some sizeable runtime machinery to be able to handle that.

\subsection{Communication}

For communication, we use multi-producer-single-consumer\cite{StdSyncMpsc} (\texttt{mpsc} for short)
channels. Our choice for this type of channel is restricted by the fact that this is the only one
available in the standard library and we preferred keeping the implementation dependency-free.
\ref{figure:figures/design.png} depicted the communication medium for the threads as something
similar to a bus, where all threads can directly communicate with all other threads. In reality, the
layout of the channels is a bit more complicated.

The process of using Rust's \texttt{mpsc} channels is such that each channel has one receiver and one
producer handle, and the producer handle can be freely copied into different threads (while the
receiver cannot be copied around -- this is ensured by the Rust's compile time checks). With this
approach, each thread has an \texttt{mpsc} channel and keeps the receiving handle to itself,
while giving a copy of the producer handle to all other threads.

Listing \ref{lst:mpsc} demonstrates this process.

\begin{lstlisting}[caption={How Channels Allow Communication Between Threads (with slight simplification)}\label{lst:mpsc}]
// In the local thread.
let (producer, receiver) = std::mpsc::channel();

// spawn a new thread.
std::thread::spawn(|| {
	// this scope is local to a new thread.
	let local_producer = producer.clone();
	// note the cloned handle ----^^^^^^^

	local_producer.send("thread1");
});

// spawn another new thread.
std::thread::spawn(|| {
	// this scope is local to a new thread.
	let another_local_producer = producer.clone();
	// note the cloned handle -----------^^^^^^^

	another_local_producer.send("thread2");
});

// check incoming messages.
while let Ok(msg) = receiver.rcv() {
	// do something with `msg`.
}
\end{lstlisting}


\section{Example Runtime: Balances} \label{chap_impl:sec:balances}

Next, we demonstrate an example runtime to help the readers get familair with the context of the implementation and how the final outcome looks like. Our final implementation supports
multiple \textit{modules} within the blockchain runtime, each having their own specific business
logic. The simplest example of such module is a \texttt{balances} module that takes care of storing
the balance of some accounts and allows the transfer of tokens between them. We now enumerate some
of the important bits of code involved in these modules.

First, there needs to be a \texttt{struct} to store the balance of a single account, which we named
\texttt{AccountBalance} - see listing~\ref{lst:balance_struct} for the full definition.

\begin{lstlisting}[caption={Balance Strict}\label{lst:balance_struct}]
/// The amount of balance that a certain account.
#[derive(Debug, Clone, Default, Eq, PartialEq, Encode, Decode)]
pub struct AccountBalance {
	/// The amount that is free and allowed to be transferred out.
	free: u128,
	/// The amount that is reserved, potentially because of the balance being used in other modules.
	reserved: u128,
}
\end{lstlisting}

\begin{remark}
	The support for \texttt{reserved} balances allows more sophisticated testing, further details of
	which are outside the scope of this chapter.
\end{remark}

The state layout of this modules is simple: there needs to be one mapping, with the key being an
account identifier and the value being \texttt{AccountBalance}, as in listing
\ref{lst:balance_struct}.

\begin{lstlisting}[caption={State Layout of the Balances Module}\label{lst:balance_state}]
decl_storage_map!(
	// Auxillary name assigned to the storage struct.
	BalanceOf,
	// Auxillary name used in key hashing.
	"balance_of",
	// Key type.
	AccountId,
	// Value type.
	AccountBalance,
);
\end{lstlisting}

Note that the statement in listing~\ref{lst:balance_state} is a macro (denoted by the \texttt{!} notation), meaning that it
generates a substantial amount of code at compile time. Most of this code deals with functions for
generating the key of a specific account's balance, namely the hashing and concatenation method
explained in \ref{chap:impl:subsec:state_and_concurrent_map}. To recap, the code generated by this
macro means: the final state key of an any account in the storage is computed as:
\texttt{"balances:balance\_of".hash() + account.hash()}.

Overall, the inline documentation of the listing should be clear in delivering: There exists a state
\textit{mapping} from \texttt{AccountId} to \texttt{AccountBalance}". We have already seen what
\texttt{AccountBalance} is, and \texttt{AccountId} is an alias for --no surprise-- a public key.

Finally, we can look at the only public transaction that can be executed in this module: a transfer
of some tokens from one account to another one, presented in listing~\ref{lst:balance_tx}.

\begin{lstlisting}[caption={Transfer Transaction}\label{lst:balance_tx}]
#[access = (|origin| vec![
	<BalanceOf<R>>::key_for(origin),
	<BalanceOf<R>>::key_for(dest)
])]
fn transfer(runtime, origin, dest: AccountId, value: Balance) {
	// read the balance of the origin.
	let mut old_balance = BalanceOf::read(runtime, origin).or_forward()?;

	if let Some(remaining) = old_balance.free.checked_sub(value) {
		// origin has enough balance. Continue.
		old_balance.free = remaining;

		// new balance of the origin.
		BalanceOf::write(runtime, origin, old_balance).unwrap()

		// new balance of the destination.
		BalanceOf::mutate(runtime, dest, |old| old.free += value).or_orphan()?;

		Ok(())
	} else {
		Err(DispatchError::LogicError("Does not have enough funds."))}
	}
\end{lstlisting}

The logic of the transaction should be straightforward: (1) read the origin's balance; (2) if they have
enough free balance, update the balance of the origin and destination, (3) Else, return an error.

The more interesting bit is in the lines 1-4 of listing~\ref{lst:balance_tx}, where our long promised \texttt{access} macro is being
used in action. The interpretation of the macro is basically as follows: this transaction will (most
likely\footnote{Recall that the \texttt{access} macro was supposed to be a best effort guess.})
access two state keys, the balance of the origin and the balance of the destination.

\section{Generic Distributor}

Another note-worthy detail of the implementation is the distributor component. Recall that the
distributor is responsible for tagging each transaction in the transaction queue with the identifier
of one thread. We emphasize that we leave this detail \textit{generic} in our implementation, similar
to its position in chapter \ref{chap:approach}. This means that there is no concrete implementation
of a distributor in our system. Instead, any function that satisfies a certain requirement can be
plugged in and used as the distributor. The main detail to remember is that we \textit{prefer} the
distributor to use the hints provided by the \texttt{access} macro. because (see
chapter \ref{chap:approach}) such a distributor will be a lot more effective at preventing
transaction forwarding.

Two examples of distributor implementation are as follows:

\begin{itemize}
	\item \textbf{Round Robin}: This distributor simply ignores the \texttt{access} macro and
	assigns transactions to threads one at a time, in a sequence.
	\item \textbf{Connected Components}\cite{nuutilaFindingStronglyConnected1994}: This graph
	processing algorithm is the exact opposite end of the spectrum compared to Round Robin, meaning
	that it \textit{heavily} takes the \texttt{access} macro into account.

	Specifically: all transactions provide a list of state keys
	that they might access during their execution, via the \texttt{access} macro. This distributor
	builds a graph of transactions and state keys, where each transaction has an edge to all the keys
	that it might access. Therefore, two transactions that are likely to access the same state key
	end up being \textit{connected}, because they both have an edge to that key. The connected
	component, as the name suggests, identifies these connected transactions. Every group of
	transactions that access a common set of state keys are grouped as a component. Once all
	components are identified, they are distributed among threads as evenly as possible. In essence,
	keeping a full component as a unit of work distribution ensures that transactions that might conflict will end up being sent to
	the same thread, effectively minimizing forwarded and orphaned transactions.
\end{itemize}

\section{Bonus: Optimizing Orphans} \label{chap_impl:sec:opt_orph}

A closer look at \ref{lst:balance_tx} reveals that the first state access error is being handled by
\texttt{.or\_forward()?} and the second one with \texttt{.or\_orphan()?}. The reason for this is
quite interesting: Recall from \ref{chap_approach:subsec:comment_baseline} that an orphan
transaction is basically one that needs access to state keys that are tainted by at least two
different threads. From this, we can realize: if a transaction successfully accesses any state key,
this means that no other thread can execute this transaction, because the key associated with that
first state access is already tainted. Therefore, we can conclude the a massive simplification: Only
if the first state access of a transaction fails it will will to be forwarded. Any further failure
is simply an orphan right off the bat.

\section{Bonus: Taintable State} \label{chap_impl:sec:bonus}

So far, all the mentioned details of this chapter are somewhat necessary to be
able to comprehend the evaluation. Conversely, this last section is optional, and explains some of the
details of the Taintable state implementation. This explanation does assume even more Rust
knowledge from the reader.

Let us recap the situation in the state \texttt{HashMap}. We know that the state is basically a
\texttt{HashMap<Vec<u8>, Vec<u8>>}, and we already know how the key is constructed (hashing and
concatenation of some prefixes and values). Two questions arise:

\begin{itemize}
	\item How is the \textit{value} constructed? For example the \texttt{decl\_storage\_map!()} in
	the balances module mapped an \texttt{AccountId} to an \texttt{AccountBalance}. How exactly is
	the \texttt{AccountBalance} encoded to \texttt{Vec<u8>}?
	\item We claimed that the state is almost wait-free. How can the underlying implementation
	support this? As noted, our tainting logic is very special to our use case, and the
	implementation of a typical concurrent \texttt{HashMap} will not be a good inspiration for us,
	because it would have different waiting semantics.
\end{itemize}

The answers are actually connected and are provided together.

Recall that our initial proposal was for the state to be wait-free, meaning that any access to the state
would succeed or fail immediately. We can solidify this idea as such: \textit{only the first} access
to each key of the state might incur some waiting for other threads; thereafter, all operations
are wait-free. It should be clear why we can obtain this property: if a key is accessed just
once, it is tainted, and, therefore, all further accesses can be immediately responded based on that
taint value. Specifically, a following access from the owner thread succeeds, while a following access from
another thread fails. It is, unfortunately,
impossible to allow \textit{all} accesses to be wait-free, because a write operation to a key that
had not existed before could trigger a resize of the hashmap.

\begin{remark}
	Now it is clear why we titled our design in \ref{chap_desgin:sec:design} \textit{almost}
	wait-free.
\end{remark}

Next, we discuss how this behoviour is implemented. First, we explain what the actual value stored in the
state is. Listing \ref{lst:state_value} shows the data type stored in the state. Note that the generic struct has
separate \texttt{taint} and \texttt{data} fields. Moreover, the taint is \textit{optional}, meaning
that it may or may not exist (denoted by \texttt{Option<\_>}).

\begin{lstlisting}[caption={The state value type}\label{lst:state_value}]
pub struct StateValue<V, T> {
	/// The data itself.
	data: RefCell<V>,
	/// The taint associated with the data.
	taint: Option<T>,
}
\end{lstlisting}

The final state type is defined in listing~\ref{lst:state_type}. Note that the key type, the value type, and the taint type are all left out generic.

\begin{lstlisting}[caption={The final \textbf{(generic}) state type}\ref{lst:state_type}]
pub type StateType<K, V, T> = HashMap<K, StateValue<V, T>>;
\end{lstlisting}

Now we can re-iterate on listing \ref{lst:maps} and correct it, leading to the implementation in listing~\ref{lst:state_final}.
\begin{lstlisting}[caption={The final (\textbf{concrete}) state type}\label{lst:state_final}]
pub type Key = Vec<u8>;
pub type Value = Vec<u8>;
pub type ThreadId = u8;

pub type State = StateType<Key, Value, ThreadId>;
\end{lstlisting}

To recap, the \texttt{Key = Vec<u8>} is computed from hashing and concatenation, and the
\texttt{Value = Vec<u8>} is the binary encoding of any data type that we may store in the state - for
example \texttt{u128} for the case of balances (as seen in \ref{lst:balance_struct}).

Finally, we can demonstrate the locking procedure. To implement the state, we wrap a
\texttt{StateType} in a \texttt{RwLock} inside a new struct.
This struct will then implement appropriate methods to allow the runtime to access the state - see listing~\ref{lst:struct_taint_state}.

\begin{lstlisting}[caption={The wrapper for \texttt{StateType}}\label{lst:struct_taint_state}]
/// Public interface of a state database. It could in principle use any backend.
pub trait GenericState<K, V, T> {
	fn read(&self, key: &K, current: T) -> Result<V, T>;
	fn write(&self, key: &K, value: V, current: T) -> Result<(), T>;
	fn mutate(&self, key: &K, update: impl Fn(&mut V) -> (), current: T) -> Result<(), T>;
}

/// A struct that implements `GenericState`.
///
/// This implements the taintable struct. Each access will try and taint that state key. Any further
/// access from other threads will not be allowed.
///
/// This is a highly concurrent implementation. Locking is scarce.
#[derive(Debug, Default)]
pub struct TaintState<K: KeyT, V: ValueT, T: TaintT> {
	backend: RwLock<StateType<K, V, T>>,
}

impl<K, V, T> GenericState<K, V, T> for TaintState<K, V, T> {
	// implementation
}
\end{lstlisting}

The most interesting piece of code omitted in \ref{lst:struct_taint_state} is the \texttt{//
implementation} part of \texttt{GenericState} for \texttt{TaintState}: this where we define how and
when we use the \texttt{RwLock} of \texttt{backend} in \texttt{TaintState}.
%We explain this verbally instead of providing more code snippets.
Specifically, each access to backend will first acquire a read lock (which is not blocking,
and other threads can also access it at the same time).

\begin{itemize}
	\item If the key is already tainted, the state operations will fail or succeed trivially: all
	non-owner threads will receive an error with the thread identifier of the owner, and  all owner
	operations will succeed. Moreover, note that the inner data in the map itself is wrapped in a
	\texttt{RefCell}, which allows interior mutability\cite{RefCellInteriorMutability}. In essence,
	this means that the \textit{owner} thread can manipulate the data only with acquiring a read lock.
	Our tainting logic and rules ensure that this will not lead to any race conditions.
	\item If a key is not already tainted (or non-existant in the inner \texttt{HashMap}), then the
	thread proceeds by trying to acquire a write lock. This is needed, because this operation is
	going to alter the \texttt{taint} field of a \texttt{StateValue}, and all other threads need to
	be blocked. Once the taint has been updated, the write access is immediately dropped, so that
	all other threads can proceed.
\end{itemize}

One final important remark is of interest. We mentioned Rust's memory safety with much confidence in
earlier sections of this chapter, claiming that it can prevent memory errors in compile time.
Nonetheless, we see that using a \texttt{RefCell}, we can alter some data, even when shared between
threads, without any synchronization. How is this possible?

The key is that Rust does allow such operations in \texttt{unsafe} mode. The "unsafe mode" of Rust is
based on a contract between the programmer and the compiler, where the programmer manually
testifies that certain operations are memory safe, and need not be checked by the compiler. In
some sense, unsafe Rust is like a fallback to \texttt{C}, where memory can be arbitrarily accessed,
allocated, de-allocated and such.

In our example, we have such a contract with the compiler as well. We know that:

\begin{itemize}
	\item If all threads access the \textit{taint} with a \textbf{write} lock, and
	\item If all threads access the \textit{data} after checking the taint, via a \textbf{read} lock.
\end{itemize}

All race conditions are resolved, thus no compiler checks are needed. Indeed, this contract needs to
be delivered to the compiler by a single \texttt{unsafe} block of code in our state
implementation - listing~\ref{lst:unsafe}.
Here, \texttt{Sync} is a trait to mark data types that are safe to be shared between threads. As the name
recommends, it is a marker trait with no functions. \texttt{RefCell} is not \texttt{Sync} by
default, because it allows arbitrary interior mutability. Because our \texttt{StateValue} contains
a \texttt{RefCell}, it is also not \texttt{Sync} be default. The statement in this listing, which needs to be
\texttt{unsafe}, tells the compiler to make an exception in this case and allow \texttt{StateValue}
to be used in a multi threaded context. This allows us to wrap the \texttt{TaintState} in an
\texttt{Arc} (atomic reference counted pointer) and share it between threads.


\begin{lstlisting}[caption={Unsafe Implementation in State}\label{lst:unsafe}]
unsafe impl<V, T> Sync for StateValue<V, T> {}
\end{lstlisting}




