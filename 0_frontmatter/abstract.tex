\begin{abstracts}

Blockchains have a two-sided reputation: they are praised for disrupting some of our institutions
through innovative technology for good, yet notorious for being slow and expensive to use. In this
work, we tackle this issue with concurrency, yet we aim to take a radically different approach by
valuing \textit{simplicity}. We embrace the simplicity through two steps: first, we formulate a
simple runtime mechanism to deal with conflicts called \textit{concurrency delegation}. This method
is much simpler and has less overhead, particularly in scenarios where conflicting transactions are
relatively rare. Moreover, to reduce the number of conflicting transactions, we propose using static
annotations attached to each transaction provided by the programmer. These annotations are
pseudo-static: they are static with respect to the lifetime of the transaction, and therefore are
free to use information such as the origin and parameters of the transaction. We propose a
distributor component in our system that can use the output of the pseudo-static annotations and use
them to effectively distribute transactions between threads in the \textit{least}-conflicting way.
We name the outcome of a system combining concurrency delegation and pseudo-static annotations as
SonicChain. We evaluate SonicChain for both validation and authoring tasks against a common workload
in blockchains, namely, balance transfers, and observe that it performs expectedly well while
introducing very little overhead and additional complexity to the system.

\end{abstracts}
