\chapter*{Prelude: Web 3.0}

\begin{chapquote}{Web3 Summit 2019 - Berlin}
	My wife asked me why I spoke so softly in the house. I said I was afraid Mark Zuckerberg was
	listening!

	She laughed. I laughed. Alexa laughed. Siri laughed.
\end{chapquote}

I want to briefly recall how I got into the blockchain ecosystem, and how it turned out to be much
more serious and important than what I thought, and why I think you should think the same way. I
will be brief though, there is a lot more to read in this thesis.

I started developing blockchains in 2019, and at first I always admittedly said to my friends and
colleagues that I am interested in it from an engineering perspective. The only other facade that I
knew to blockchains was its monetary aspect. In short, blockchains were two things to me: Complex
distributed systems, and getting rich overnight. I tried to convince myself that I am only
interested in the former.

Soon after, through the works, papers, talks of very visionary people, I learned about the third,
perhaps the most important facade of blockchains: That it is merely a tool in a much larger
ecosystem of ideas and beliefs about how we collaborate, perform logistics, and live as a society.
Different people have drawn different ideas here; I will stick to just one that I find the most
relevant: Web 3.0.

Let us look back: web 1.0 was the static web of the '90s and the early years of the 20th century. It
had limited interaction, and only served to show static information. The early web and its
underlying protocols were designed for government and educational purposes. Trust is an absolutely
vital asset in web 1.0.

Then came along Mark Zucker... I mean web 2.0. It was dynamic, it could do much more and started
offering way more. We starter building financial and institutional systems on top of it. Except, we
forgot to update any of the protocols. Web 2.0 expanded very quick and very fast, and brought us
here: We can do many great things with web 2.0, it is great. But the data and sovereignty of the
user is entirely at the mercy of giant tech corporations. In a sense, we let web 2.0 expand
exponentially, without thinking that its underlying protocols -- the \textit{trust} -- is still
there, it is becoming more dangerous every day.

It was quite a shock to me to learn that blockchain-enthusiasts are as excited to talk about Satoshi
Nakatomo and the daily bitcoin price, as they are to talk about the Snowden revolution, privacy,
tracking-based advertising and its glooming effects, and democracy. And this is where web 3.0 comes
into play. Web 3.0, essentially, is supposed to eliminate the \textit{trust} embedded in the web 2.0
protocols, allowing users to interact with one another in a trust-less, decentralized, and more
secure manner. And in this picture, blockchain is really just the tip of the iceberg, just one piece
of the puzzle. A much wider army of tools, protocols and science is needed to develop and build web
3.0. This ranges from bare-bone sciences such as probability and cryptography, to social matters
such as privacy, identity, and transparent governance of protocols, all the way to the typical
stuff: user-friendly interfaces and APIs for other developers to build upon.

I learned over time that blockchains are not as impressive as they seem. Rather it is the underlying
aspiration that makes it more interesting. I hope you feel similar someday.
