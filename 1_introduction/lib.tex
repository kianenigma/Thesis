\chapter{Introduction} \label{chap:intoroduction}

% the code below specifies where the figures are stored
\ifpdf
    \graphicspath{{1_introduction/figures/PNG/}{1_introduction/figures/PDF/}{1_introduction/figures/}}
\else
    \graphicspath{{1_introduction/figures/EPS/}{1_introduction/figures/}}
\fi

\begin{chapquote}{Unknown.}
``If Bitcoin was the calculator, Ethereum was the ENIAC\footnote{the first generation computer
developed in 1944. It fills a 20-foot by 40-foot room and has 18,000 vacuum tubes.}. It is
expensive, slow and hard to work with. The challenge of today is to build the \textbf{commodity,
accessible and performant} computer.''
\end{chapquote}

\todo[inline]{

Meeting notes:

Why is this problem interesting??

mostly short. Start with requirements (correctness).

Chapter 3 should be roughly design and impl details. and it should promise some goals that you meet
in chapter 4.

Design in Notion is final. Write a set of requirements and explain \textbf{why} the design respects
it.

related work: say where you fit in the literature overlay. No need to brag and\textbf{compare with
others}.

experiment setup should prove the goal, not only speedup etc.}


Blockchains are indeed an interesting topic in 2020. Many believe that it is a revolutionary
technology that will shape our future societies, much like the internet and how it has impacted many
aspect of how we live in the last few decades \cite{will_blockchain_be_big_deal}. Moreover, they are
highly sophisticated and inter-disciplinary software artifacts, achieving high levels of
decentralization and security, which was deemed impossible so far. To the contrary, some people
skeptically see them as controversial, or merely a "hyped hoax", and doubt that they will ever
deliver much real value to the world.

In a very broad term, a blockchain is a tamper-proof, append-only ledger that is being maintained in
a decentralized fashion, and can only be updated once everyone agrees upon that change as a bundle
of transactions. This bundle of transactions is called a \textbf{block}. Once this block is agreed
upon, it is appended (aka. \textit{chained}) to the ledger, hence the term block-\textit{chain}.
Moreover, the ledger itself is public and openly accessible to anyone. This means that everyone can
verify and check the final state of the ledger and all the transactions and blocks in its past that
lead to this particular ledger state, to verify everything. At the same time asymmetric cryptography
is the core identity indicator that one needs to interact with the chain, meaning that your personal
identity can stay private in principle, if that is desired based on the circumstance.

In some sense, blockchains are revolutionary because they remove the need for \textit{trust}, and
release it from the control of one entity (i.e. a single bank or institute), by encoding it as a
self-sovereign decentralized software. Our institutions are built upon the idea that they manage
people's assets, matters and belongings, and they ensure veracity, because we trust in them. In
short, they have \textbf{authority}. Of course, this model could work well in principle, but it
suffers from the same problem as some software do: it is a \textbf{single point of failure}. A
corrupt authority is just as destructive as a flaky single point of failure in a software is.
Blockchain attempts to resolve this by deploying software (i.e. itself) in a transparent and
decentralized manner, in which everyone's privacy is respected, whilst at the same time everyone can
still check the veracity of the ledger, and the software itself. In other words, no single entity
should be able to have any control over the system.

Now, all of these great properties don't come for cheap. Blockchains are extremely complicated
pieces of software and they require a great deal of expertise to be written correctly. Moreover,
many of the machinery used to create this \textit{decentralized} and \textit{public} append-only
ledger, requires synchronization, serialization, or generally other procedures that are slow and
reduce the throughput at which the chain can process transactions. This, to some extent, contributes
to the skepticism mentioned in at the beginning of this chapter. For example, Bitcoin, one of the
famous deployed blockchains to date, consumes a lot of resources to operate, and cannot exceed a
transaction throughput of more than half a dozen transactions per second, yet at the same time it
consumes a \textit{lot} of computation power to operate correctly \cite{security_of_bitcoin}.

Hence, we see it as in important goal to investigate the possibilities through which a blockchain
system can be enhanced to perform \textit{faster}, and more \textit{efficiently}.

\section{Research Question} \label{chap_intro:sec:resarch_q}

Based on the mentioned scenario, we can settle on the basic goal of improving the
\textit{performance} and \textit{efficiency} of a blockchain system. There are numerous ways to
achieve this goal, ranging from zooming into the details such as concurrency within some components
of the software, all the way back to tuning the network parameters and implementations. In this
work, we precisely focus on the former, enabling concurrency within transactions that are processed
and then appended to the ledger. This approach is better compared with the other alternatives at the
end of chapter \ref{chap:background}. We also mention in the same chapter why each of these
approaches could have their own merit and value, and how they differ with one another.

 All in all, we formulate the following as our research questions:

 \begin{enumerate}
     \item What approaches exist to deploy concurrency methods in the realm of blockchains, when
     seen as a generic decentralized transaction processing systems?
     \item How deficient and reasonable are each these approaches, given that different blockchains
	 can have radically different transaction types and concurrency requirements.
	 \item \todo[inline]{potentially more, write and rephrase as you proceed.}
 \end{enumerate}

\section{Rest of this work}

The first half of this work leads the way toward our proposed approach to solving the aforementioned
research question. \todo[inline]{should I say in this work or this thesis or our work or our thesis?}. In
chapter \ref{chap:background}, we dive deep into the two pillars of background knowledge that we
need to answer the above questions: blockchains and concurrency. First, we will clarify what exactly
a blockchain is in section. Then, having known this, we can have a 100-feet view into a blockchain
system and see what are the broad terms ways in which its speed can be improved in section
\ref{chap_bg:sec:ways_to_speedup} . Then, in chapter \ref{chap:design_and_impl} we will define the
requirements of the system and consequent design goals. This is followed by our proposed design in
section \ref{chap_desgin:sec:design}. Finally, we provide some implementation details in
\ref{chap:design_and_impl}.

In the second part of this work, we put our solution to action and observe the outcome. Chapter
\ref{chap:bench_analysis} will be devoted to the benchmarks and experiments through which we
evaluate whether our research goals have been met. An important detail of this chapter is the data
set, explained in section \ref{chap_b&a:sec:data_set}. In chapter \ref{chap:related} we compare our
work with other similar research in the domain and see how it fits in the scientific literature.
Finally, we conclude and propose future work in chapter \ref{chap:conclusion}.

